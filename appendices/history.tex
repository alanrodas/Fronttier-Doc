\section{Historia de los sistemas empresariales}
\label{appendix:history}

Este apartado dará una breve muestra de las transformaciones que ha
sufrido la industria de desarrollo de sistemas en los últimos tiempos. Se verá 
también como estas transformaciones han llevado a que los sistemas web sean hoy 
en día una de las opciones más utilizadas en el ámbito empresarial.\\

\subsection{Estado anterior a los sistemas web}
\label{subsec:history:about_web:previous_pc}

En las décadas de 1960 y 1970, comienza en el mundo un proceso lento pero
incremental de computarización de la información. Las computadoras dejan de ser
\quoted{juguetes} científicos para pasar a ser complejas maquinarias con
verdadera utilidad practica en distintas industrias.\\
Durante esos años varias empresas comienzan a \emph{informatizarse}. Bancos,
petroleras y otros compran grandes computadoras capaces de almacenar toda su
información o procesar complejos cálculos matemáticos en poco tiempo.\\
En este contexto, múltiples empresas como \emph{IBM}, \emph{RCA}, y 
\emph{General Electric}, comenzaron a fabricar gigantescas computadoras de
precio cada vez más económico. Estas maquinas evolucionarían poco a poco hasta
convertirse en lo que hoy se conocen como \mainframes.\\
En este sentido, \citefullauthor{Stephens:2008:BOOK} en su libro
\citetitle{Stephens:2008:BOOK} \citeyearsq{Stephens:2008:BOOK} describe a un
\mainframe como una computadora muy grande, con una base de datos de alto
rendimiento y a la cual se accede desde una terminal remota. Básicamente, una
maquina \quoted{tonta}, sin mucha otra finalidad más que la de conectarse al
servidor central para realizar peticiones de datos y mostrarle los resultados
al usuario.\\
\citeauthor{Stephens:2008:BOOK} también destaca los beneficios de los
\mainframes por sobre una computadora personal a nivel empresarial. Las
características que menciona son:
\begin{description}[
	font=$\bullet$\enskip,
	leftmargin={\parindent*4},
	labelindent=\parindent]
	\setlength{\itemsep}{1pt}
	\setlength{\parskip}{0pt}
	\setlength{\parsep}{0pt}
	\item[Integridad de datos:] Los datos \textsc{deben} ser correctos.
	\item[Rendimiento:] Se debe procesar gran cantidad de datos.
	\item[Respuesta:] El procesamiento debe ser inmediato.
	\item[Recuperación ante desastres:] En caso de un fallo, se debe volver a
										estar operativo inmediatamente.
	\item[Usabilidad:] Debe hacer lo que se requiere, cuando se requiere.
	\item[Confiabilidad:] No debe fallar y siempre debe estar disponible.
	\item[Auditoria:] Ser capaz de saber quien realizó qué acción en el equipo.
	\item[Seguridad:] Solo aquellos que pueden realizar una acción pueden
					  hacerlo.
\end{description}
Estas características son todavía buscadas en los grandes \mainframes de la
actualidad, y resultan fundamentales en emprendimientos críticos.\\
\citefullauthor{Elliot:2008:BOOK} ha descripto que las tendencias de la
tecnología utilizada, no obedecen solamente a motivaciones económicas y
operativas, sino a un complejo entramado de relaciones entre empresas que 
comparten una visión utópica sobre la tecnología en cuestión
\citesq[pag 3]{Elliot:2008:BOOK}. Así, las empresas comercializadoras de
\mainframes enarbolando como bandera principal los beneficios planteados por
\citeauthor{Stephens:2008:BOOK}, lograron generar una visión utópica en las
empresas usuarias para posicionar la tecnología dominante durante largos años.
Es por esto que los \mainframes siguen teniendo hoy en día un lugar en el
mercado.\\
 
\subsection{Aparición de las computadoras personales}
\label{subsec:history:about_web:pc_era}

A partir de fines de los 70`s y comienzos de los 80`s, las \acp{pc}
comienzan a ser cada vez más accesibles \citesq[cap 4]{Allan:2001:BOOK}.
La liberación de \emph{ARPANET} por parte de \acs{darpa} en 1983 da nacimiento
a \internet. Estos dos cambios suponen un quiebre en las tecnologías de uso
empresarial.\\
El \ac{pc}, sumado a la conectividad y nuevas tecnologías tanto en sistemas
como en lenguajes de programación llevaron a que los \mainframes comenzaran a
perder terreno.\\
La posibilidad de ejecutar programas en cada \ac{pc} dio lugar a nuevas
arquitecturas \clientserver. Ahora los usuarios corrían los programas en su 
propio equipo, permitiendo aumentar la velocidad de respuesta. El programa se
conectaba a \internet y sincronizaba información con un servidor central solo 
cuando fuera necesario.\\
Esta arquitectura permitiría una más rápida evolución del software por sobre la 
opción de tener el sistema en un \mainframe, como explican
\citefullauthor{Duncan:1996:ARTICLE} en su artículo 
\citetitle{Duncan:1996:ARTICLE}\period Como contrapartida estos cambios suponen
un alto costo de mantenimiento, y en caso de poseer equipos con distintos
sistemas operativos, un gasto extra en el desarrollo. Independientemente de los 
costos, el factor más incidente en el cambio es la necesidad de mantener el 
sistema al día con las necesidades de la empresa, señalan 
\citeauthor{Duncan:1996:ARTICLE}.

\subsection{Navegadores y primeros sistemas web}
\label{subsec:history:about_web:web}

Tras la aparición de los navegadores web (como \emph{Netscape} e
\emph{Internet Explorer}) a comienzos de 1990 se comenzaron a desarrollar los 
primeros sistemas web.\\
Los mismos consistían básicamente en documentos dinámicos, generados con 
información tomada de una base de datos. Gran influencia tuvieron en este tipo 
de soluciones la creación de lenguajes de programación y tecnologías pensadas 
para generar documentos \html dinámicamente, como \emph{PHP}, \emph{ASP} y los 
\emph{Servlets} de \java \citesq[pag 2]{Hunter:2001:BOOK}. La posibilidad de 
generar \emph{aplicaciones} que corran en el navegador solucionó las
problemáticas asociadas a el mantenimiento de los sistemas y al desarrollo para 
múltiples sistemas operativos. Sin embargo, la naturaleza de \html limitaba la 
capacidad de las aplicaciones a sencillos formularios, botones y enlaces.\\
Estas limitaciones hicieron que no fuera sino hasta con la aparición de las 
primeras tecnologías complementarias a \html que las aplicaciones web 
comenzarían a cobrar relevancia, dando lugar a las \ac{ria}.

\subsection{Rich Internet Applications}
\label{subsec:history:about_web:rias}

Las \glspl{ria} surgen para compensar aquellas faltantes que presentaban las 
aplicaciones web frente a las tradicionales de \clientserver.\\
En un primer lugar software en forma de complementos (\plugins) que permitían
correr algún lenguaje especial en los navegadores fueron creados.\\
Las \ria comenzaron a copar los mercados empresariales, ya que permitían ahorrar
en mantenimiento y desarrollo a la vez que brindaban la usabilidad que se 
demandaba de un sistema empresarial.\\
Cada vez más soluciones comenzaron a surgir y los \plugins privativos 
comenzaron a ser un problema para los desarrolladores. Esto llevaría a que se 
desarrollaran con fuerza \css y \js, tecnologías libres que permitían crear 
\rias, disponibles en cualquier navegador sin necesidad de instalar software 
adicional.\\
Así, en los últimos años, y gracias también al auge de los smartphones y 
tablets, incapaces de correr los anteriormente mencionados \plugins, el llamado 
\htmlv pasaría a transformarse en el estándar para el desarrollo de \rias
\citesq{David:2013:BOOK}.\\

\subsection{Conclusiones}
\label{subsec:history:conclusions}

Los sistemas web actuales permiten la robustez esperada de los primeros sistemas
\mainframe, pero sin los altos costos de adquisición y mantenimiento de los 
mismos. A su vez, las \rias ofrecen la capacidad de tener aplicaciones con un 
alto grado de usabilidad. Además, la gran cantidad de \frameworks capaces de 
generar rápidamente sistemas web con una interfaz completamente 
realizada en \htmlv que han aparecido en los últimos años, permiten reducir los 
tiempos de desarrollo. Junto con las plataformas como servicio, como 
\emph{Heroku} o \emph{Rackspace} reducen significativamente la puesta en 
producción de los sistemas.\\
Todas estas características han llevado a que las aplicaciones web con una 
\viewtier hecha en \htmlv se hayan transformado en la nueva visión utopica 
señalada por \citeauthor{Elliot:2008:BOOK}.Así, este tipo de desarrollos 
seguirán siendo la norma durante los años venideros.
