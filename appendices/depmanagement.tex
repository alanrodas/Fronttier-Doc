\section{Manejo de distintos tipos de \dependencies}
\label{appendix:depmgmnt}

En esta sección se explicará como se manejan las \dependencies en las distintas
capas de software. Se mostrará como la solución propuesta en este documento se adapta a técnicas ya probadas.\\

\subsection{Dependencias manejadas de la \logictier}
\label{subsec:depmgmnt:jvm_dev:logic_dependencies}

En el ámbito del manejo de \dependencies en la \logictier la herramienta \maven
de la fundación \apache se ha posicionado como el estándar de facto\footnote{
	Si bien \apache \maven es en realidad no solo un \depmgr sino un 
	administrador de proyectos, provee la funcionalidad antes mencionada y
	la misma se ha vuelto un estándar.
} \citesq{Sonatype:2008:BOOK}. Otro de los proyectos que se han posicionado en 
esta área es \apache \ivy, que se concentra solamente en el manejo de 
\dependencies.\\
\maven hace uso de un \conffile escrito en \xml, archivo comúnmente denominado 
POM, ya que el nombre del archivo es pom.xml. El 
\coderef{code:depmgmnt:jvm:maven_pom} muestra un ejemplo de este tipo de 
\conffile. \ivy utiliza también archivos \xml, pero estos son más compactos que 
los de \maven. El \coderef{code:depmgmnt:jvm:ivy_module} es un ejemplo del mismo.\\

\begin{listing}[htb]
\begin{minted}[frame=lines]{xml}
<project>
  ...
  <dependencies>
    <dependency>
      <groupId>group-a</groupId>
      <artifactId>artifact-a</artifactId>
      <version>1.0</version>
    </dependency>
    <dependency>
      <groupId>group-a</groupId>
      <artifactId>artifact-b</artifactId>
      <version>1.0</version>
    </dependency>
  </dependencies>
</project>
\end{minted}
\caption{Configuración de \dependencies en \maven mediante archivo POM}
\label{code:depmgmnt:jvm:maven_pom}
\end{listing}

\begin{listing}[htb]
\begin{minted}[frame=lines]{xml}
<ivy-module version="2.0">
  <info organisation="org.apache" module="hello-ivy"/>
  <dependencies>
    <dependency org="group-a" name="artifact-a" rev="1.0"/>
    <dependency org="group-b" name="artifact-b" rev="1.0"/>
  </dependencies>
</ivy-module>
\end{minted}
\caption{Archivo de configuración de Ivy}
\label{code:depmgmnt:jvm:ivy_module}
\end{listing}

Con la aparición de nuevos lenguajes para la \jvm comenzaron a surgir nuevos
\depmgrs, como Grape o Gradle para \groovy, \sbt para \scala, Leiningen para 
\clojure, y \apache Buildr como propuesta multilenguaje. Cada uno de ellos 
utiliza un \conffile con distinta sintaxis
\citesq[pág. 569]{Alexander:2013:BOOK}. El 
\coderef{code:depmgmnt:jvm:sbt_module_add} muestra un ejemplo de la declaración de 
una \dependency en el \conffile de \sbt.\\

\begin{listing}[htb]
\begin{minted}[frame=lines]{scala}
libraryDependencies += "org.springframework.data" %
    "spring-data-neo4j" % "3.1.1.RELEASE"
\end{minted}
\caption{Dependencia de SpringFramework 3.1.1 para \sbt}
\label{code:depmgmnt:jvm:sbt_module_add}
\end{listing}

A pesar de la gran cantidad de formatos de \conffile, todos los \depmgrs
trabajan bajo el estándar pre-impuesto por \maven. Esto llega a un punto tal 
que el sitio \gls{mvnrepo} es el centro de referencia al momento de buscar 
dependencias. Se puede entonces apreciar como todos los \conffiles especifican
inequívocamente un paquete de código mediante tres campos clave para su 
búsqueda: el nombre de la compañía, el nombre del modulo y la versión del 
programa.\\

\subsection{Dependencias no manejadas de \viewtier}
\label{subsec:depmgmnt:jvm_dev:view_dependencies}

La mayoría de las \dependencies de la \viewtier consisten en archivos 
principalmente \css y \js (aunque también podrían ser fuentes web u otros 
recursos), los cuales son descargados manualmente a través de la página del 
creador del código (No existen repositorios centralizados como en el caso de 
las \dependencies de la \logictier). Una vez descargados, el desarrollador debe 
mover los archivos a alguna carpeta de su proyecto para \emph{instalarlo} (la 
ubicación exacta depende de la estructura del proyecto y por tanto debe ser 
recordada por el programador) para encontrarse finalmente disponibles para su 
uso en el código del usuario.\\
Los errores en esta etapa son muy comunes dado la variada cantidad de tipos de
\dependency, donde cada uno debe ser copiado en una ubicación puntual. Además 
una misma \dependency no necesariamente consiste en un único archivo (como 
sucede en la \logictier), sino en múltiples archivos de distinto formato que 
deben ser agregados al proyecto de una forma especifica. La falta de un lugar 
que agrupe estos contenidos y estandarice sus nombres, versiones, formas de 
instalación, etc. hace que sea muy difícil manejar este tipo de \dependencies.

\subsection{CDN}
\label{subsec:depmgmnt:jvm_dev:cdns}

Las \emph{Redes de Distribución de Contenido}, (\cdn) surgen con la idea de agrupar el contenido disponible online en un solo lugar\footnote{
	Su creación no responde necesariamente a procesos de manejo de dependencias
	en proyectos de desarrollo de software, sin embargo es uno de los usos
	posibles de este tipo de redes y ha sido uno de los usos más comunes.
}.\\
Así, las \cdn permiten al usuario insertar en el código \html de un proyecto, 
una clausula que importa automáticamente una dependencia desde el 
servidor de la \cdn más cercano.\\

\begin{listing}[ht]
\begin{minted}[frame=lines]{html}
<html>
  <head>
    ...
    <!-- JQuery -->
    <script
  src="//ajax.googleapis.com/ajax/libs/jquery/1.11.1/jquery.min.js">
    </script>
    <!-- JQuery Mobile -->
    <link rel="stylesheet"
  href="//ajax.googleapis.com/ajax/libs/jquerymobile/1.4.3/jquery.mobile.min.css"/>
    <script
  src="//ajax.googleapis.com/ajax/libs/jquerymobile/1.4.3/jquery.mobile.min.js">
    </script>
    <!-- JQuery UI -->
    <link rel="stylesheet" 
  href="//ajax.googleapis.com/ajax/libs/jqueryui/1.11.0/themes/smoothness/jquery-ui.css"/>
    <script
  src="//ajax.googleapis.com/ajax/libs/jqueryui/1.11.0/jquery-ui.min.js">
    </script>
  </head>
  <body>
    ...
  </body>
</html>
\end{minted}
\caption{Dependencias agregadas mediante \cdn}
\label{code:depmgmnt:cdn:cdn_deps}
\end{listing}

El \coderef{code:depmgmnt:cdn:cdn_deps} utiliza la \cdn de 
\citeauthor{GoogleCDN:ONLINE} (\citeurl{GoogleCDN:ONLINE}) una de las más 
populares y confiables junto con la de \citeauthor{MicrosoftCDN:ONLINE} 
(\citeurl{MicrosoftCDN:ONLINE}) y los sitios \citeurl{jsDelivrCDN:ONLINE} y 
\citeurl{CDNjs:ONLINE}.\\
Las \cdns permiten adicionalmente aumentar la velocidad de la aplicación que se 
desarrolla ya que los archivos de \dependency pueden ser descargados desde 
servidores más cercanos a la ubicación del usuario, y además ahorran carga de 
trabajo al servidor de la aplicación.\\
Como contrapartida, las \cdns suelen tener una gestión centralizada desde 
alguna entidad proveedora, la cual no necesariamente responde a los pedidos o requerimientos del usuario). Esto hace que sea difícil contar con las ultimas 
versiones, o con bibliotecas que los administradores del mismo decidan no 
soportar. A esto se le agrega el alto costo de mantener una \cdn por cualquier 
empresa que no posea grandes servidores a lo largo del globo.

\subsection{Dependencias manejadas de \viewtier}
\label{subsec:depmgmnt:jvm_dev:view_managed}

Otra forma de manejar \dependencies en la \viewtier consiste en utilizar el 
mismo mecanismo de \depmgrs que se utiliza en la \logictier. Esto pone solución 
a las problemáticas que acarrean las \cdns.\\
\emph{Bower} y \emph{Component} son los \depmgrs más populares para la 
\viewtier. Ambos son módulos de la plataforma \gls{node} y manejan las 
\dependencies mediante un \conffile.\\
Estas herramientas no se empaquetan de forma auto contenida, sino que requieren
contar con herramientas externas como \git o \svn en la maquina en donde se va a
utilizar \citesq{Bower:ONLINE}. Más aún, los usuarios de lenguajes que corren 
sobre la plataforma \java muchas veces carecen de acceso a la maquina física, 
por fuera de la \jvm, haciendo imposible utilizar estas opciones.\\
Adicionalmente, existen soluciones que se enfocan en un tipo especifico de 
dependencia, como \emph{RequireJS} y \emph{Browserify} que permiten manejar 
solamente archivos \js, sin utilizar archivos \conffiles, sino mediante una 
línea de código en el programa del usuario \citesq{Franko:2013:BOOK}.\\
Todas estas soluciones se presentan bastante incompletas en comparación a las
disponibles en la \viewtier. Además, al requerir herramientas que no son
independientes del sistema operativo, generan muchas complicaciones al momento
de configurar el entorno de desarrollo o de puesta en producción del sistema.
Finalmente, fuerzan al desarrollador a aprender una nueva tecnología, proceso
que lleva tiempo y esfuerzo.

\subsection{Funcionamiento de Fronttier}
\label{subsec:depmgmnt:proposal}

La herramienta propuesta funciona mediante un \conffile de forma similar a las
opciones disponibles para la \logictier. Además, los \conffile que utiliza la
herramienta, pueden poseer formatos distintos, adaptándose a los formatos que
los desarrolladores ya conocen de otras herramientas.\\
A diferencia de las opciones de manejo de dependencias que funcionan en la
\viewtier de esta forma, la herramienta propuesta no requiere herramientas
externas, y todo corre sobre la \jvm.