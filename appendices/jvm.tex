\section{Elección de la plataforma}
\label{appendix:jvm}

En esta sección se explican brevemente los factores que motivan la elección de 
la \jvm como la plataforma de trabajo. En particular son dos los motivos por 
los cuales se ha elegido esta plataforma, su popularidad y la gran cantidad de 
lenguajes que corren sobre la misma.\\

\subsection{Acerca de los lenguajes para la Java Virtual Machine (\jvm)}
\label{subsec:intro:about_jvm}

Con la popularidad de \java, comenzaron a surgir una serie de lenguajes que 
compilan a \bytecode para la \acl{jvm}. Estos lenguajes se presentan como 
alternativas para los programadores de la plataforma que buscan una opción 
"superior", en algún aspecto, a \java. \groovy, por ejemplo, se presenta como 
un lenguaje dinámico, similar a Ruby; \clojure es una extensión de Lisp 
apuntado a la Programación Concurrente; \scala es un lenguaje que presenta 
características tanto de Programación Funcional como de Programación Orientada 
a Objectos.  También existen otra numerosa cantidad de lenguajes, algunos 
experimentales, otros con una base creciente de usuarios, que corren sobre la 
plataforma. A  esto se suma una amplia cantidad de adaptaciones de lenguajes 
populares para que corran en la \jvm, como JRuby y Jython, versiones de Ruby y 
Python que compilan a \bytecode \java \citesq{Wikipedia:2014:ONLINE}.\\
Una característica interesante de estos lenguajes es que el código compilado 
suele ser compatible entre si, es decir, código escrito en \java puede ser 
ejecutado en \scala, código \groovy puede ser usado desde \clojure, 
etc.\footnote{
	Si bien mayoritariamente el código de un lenguaje para la \jvm es 
	compatible con otros lenguajes de la \jvm, ciertas características 
	especiales de estos suelen no poder ser empleadas desde otros lenguajes, 
	presentando así ciertas limitaciones de compatibilidad.
}.\\
Finalmente, podemos destacar el hecho de que hoy en día existen numerosas 
implementaciones de la \jvm, capaces de correr todos estos lenguajes en 
distintas plataformas \citesq{Wikipedia:2012:ONLINE}.\\


\subsection{Popularidad de la \jvm como plataforma}
\label{subsec:intro:jvm_growth}

Las estadísticas de análisis de popularidad de lenguajes, como la renombrada 
estadística TIOBE, demuestran que \java es uno de los lenguajes más populares 
[\citefullauthor{TIOBE:2014:ONLINE}, \citeyear{TIOBE:2014:ONLINE}]. La 
estadística \emph{"transparente"} creada por 
\citefullauthor{Montmollin:2013:ONLINE}, que publica el código de forma 
\opensource y sus reglas online, muestra al lenguaje de Oracle en la misma 
posición \citesq{Montmollin:2013:ONLINE}, mientras que la otra herramienta 
\opensource que permite medir la popularidad de los lenguajes, PyPL, de
\citefullauthor{Carbonelle:2014:ONLINE}, lo muestra en primer lugar
\citesq{Carbonelle:2014:ONLINE}.\\
Además, \citefullauthor{Kunst:2014:ONLINE} ha desarrollado un gráfico en donde 
muestra la cantidad de líneas de código en \gls{github} por sobre las menciones 
en \gls{stackoverflow}, mostrando a \java por sobre los más populares, pero 
también a los lenguajes que corren sobre la \jvm\footnote{
	Entre los lenguajes para la \jvm se encuentran \scala, \clojure, \groovy, 
	AspectJ, Ceylon, Fantom, Fortress, Frege, Gosu, Ioke, Jelly, Kotlin, Mirah, 
	Processing, X10, Xtend, Rhino, JRuby, Jython, entre otros. Aunque no todos 
	se encuentran mencionados en la gráfica.
} como de alto interés.\\