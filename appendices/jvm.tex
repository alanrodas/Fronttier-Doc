\section{JVM como plataforma: Popularidad de la JVM}
\label{appendix:jvm}

En esta sección se explican brevemente los factores que motivan la elección de 
la \jvm como la plataforma de trabajo. En particular son dos los motivos por 
los cuales se ha elegido esta plataforma, su popularidad y la gran cantidad de 
lenguajes que corren sobre la misma.\\

\subsection{Acerca de los lenguajes para la Java Virtual Machine (\jvm)}
\label{subsec:intro:about_jvm}

Con la popularidad de \java, comenzaron a surgir una serie de lenguajes que 
compilan a \bytecode para la Maquina Virtual de Java. Estos lenguajes se 
presentan como alternativas para los programadores de la plataforma que buscan 
una opción "superior", en algún aspecto, a \java. \groovy, por ejemplo, es un lenguaje dinámico, similar a Ruby; \clojure es una extensión 
de Lisp apuntado a la Programación Concurrente; \scala es un lenguaje que 
posee características tanto de Programación Funcional como de Programación 
Orientada a Objectos.  También existe otra numerosa cantidad de lenguajes, 
algunos experimentales, otros con una base creciente de usuarios, que corren 
sobre la plataforma. A  esto se suma una amplia cantidad de adaptaciones de 
lenguajes populares para que corran en la \jvm, como JRuby y Jython, versiones 
de Ruby y Python que compilan a \bytecode \java 
\citesq{Wikipedia:2014:ONLINE}.\\
Una característica interesante de estos lenguajes es que el código compilado 
suele ser compatible entre si, es decir, código escrito en \java puede ser 
ejecutado en \scala, código \groovy puede ser usado desde \clojure, 
etc.\footnote{
	Si bien mayoritariamente el código de un lenguaje para la \jvm es 
	compatible con otros lenguajes de la \jvm, ciertas características 
	especiales de estos suelen no poder ser empleadas desde otros lenguajes, 
	presentando así ciertas limitaciones de compatibilidad.
}.\\
Finalmente, se destaca el hecho de que hoy en día existen numerosas 
implementaciones de la \jvm, capaces de correr dichos lenguajes en 
distintas plataformas \citesq{Wikipedia:2012:ONLINE}.\\


\subsection{Popularidad de la \jvm como plataforma}
\label{subsec:intro:jvm_growth}

Las estadísticas de análisis de popularidad de lenguajes como la conocida TIOBE 
demuestran que \java es uno de los lenguajes más populares 
[\citefullauthor{TIOBE:2014:ONLINE}, \citeyear{TIOBE:2014:ONLINE}]. Por su 
parte, el informe \emph{"transparente"} creado por 
\citefullauthor{Montmollin:2013:ONLINE}, que publica el código de forma 
\opensource y sus reglas online, exhibe al lenguaje de Oracle en la misma 
posición \citesq{Montmollin:2013:ONLINE}, mientras que la otra herramienta 
\opensource que permite medir la popularidad de los lenguajes, PyPL, de
\citefullauthor{Carbonelle:2014:ONLINE}, lo presenta en primer lugar
\citesq{Carbonelle:2014:ONLINE}.\\
Además, \citefullauthor{Kunst:2014:ONLINE} ha desarrollado un gráfico en donde 
proyecta la cantidad de líneas de código en \emph{GitHub} por sobre las 
menciones en \emph{Stack Overflow}\footnote{
	StackOverflow es un popular sitio sobre programación, en donde usuarios
	suelen realizar preguntas a problemas puntuales de programación y otros 
	usuarios las responden.
}, mostrando a \java por sobre los más populares, pero también a los lenguajes 
que corren sobre la \jvm\footnote{
	Entre los lenguajes para la \jvm se encuentran \scala, \clojure, \groovy, 
	\emph{AspectJ}, \emph{Ceylon}, \emph{Fantom}, \emph{Fortress}, 
	\emph{Frege}, \emph{Gosu}, \emph{Ioke}, \emph{Jelly}, \emph{Kotlin}, 
	\emph{Mirah}, \emph{Processing}, \emph{X10}, \emph{Xtend}, \emph{Rhino}, 
	\emph{JRuby}, \emph{Jython}, entre otros. Aunque no todos se encuentran 
	mencionados en la gráfica.
} como de alto interés.\\