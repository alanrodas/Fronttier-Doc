\newglossaryentry{html}{
	name=HTML,
	description={HTML, siglas de \emph{HyperText Markup Language} (lenguaje de marcas de
		hipertexto), es un lenguaje de marcado para la elaboración de páginas web. Fue
		desarrollado por Tim Berners-Lee en 1991 y rápidamente se convirtió en un estándar}
}

\newglossaryentry{ria}{
	name=RIA,
	description={Una aplicación de Internet enriquecida es una aplicación que corre en el
		navegador del usuario y que permite emular complejos comportamientos y funcionalidades
		que antes solo eran disponible en sistemas de escritorio, tales como arrastras y
		soltar, ordenar elementos o crear complejos gráficos. Para esto hacen uso de distintas
		tecnologías, que pueden ser, programas que corren sobre el navegador en forma de
		complementos, o aplicaciones creadas enteramente en HTML, CSS y JavaScript}
}

\newglossaryentry{css}{
	name=CSS,
	description={CSS es un lenguaje utilizado para describir el aspecto y el formato de un
		documento escrito en lenguaje HTML o similar. Actualmente es un estandar web y
		es soportado por prácticamente todos los navegadores}
}

\newglossaryentry{js}{
	name=JavaScript,
	description={JavaScript (abreviado comúnmente JS) es un lenguaje de programación
		interpretado, orientado a objetos y basado en prototipos, utilizado principalmente del
		lado del cliente. Este lenguaje es un estándar web ya que es implementado como parte
		de prácticamente todos los navegadores web, permitiendo interactividad del usuario con
		la interfaz sin necesidad de comunicarse constantemente con el servidor}
}

\newglossaryentry{html5}{
	name=HTML5,
	description={Se conoce como HTML5 a la quinta revisión importante del lenguaje básico
		de marcado de documentos para \internet. Su nombre suele hacer referencia no solo al
		nuevo estándar, todavía experimental de este lenguaje, sino también a las tecnologías
		que lo acompañan, CSS en su versión 3, y JavaScript}
}

\newglossaryentry{jvm}{
	name=JVM,
	description={La JVM es una parte fundamental de la plataforma del lenguaje Java,
		creada originalmente por \emph{Sun Microsystems}. La maquina virtual crea una capa de
		abstracción entre el sistema operativo y el código Java compilado (\emph{Bytecode}).
		De esta forma, el código Java puede ser compilado a \bytecode una sola vez, y correrse
		en cualquier sistema que cuente con una JVM}
}

\newglossaryentry{github}{
	name=GitHub,
	description={GitHub es un popular sitio web que permite almacenar código a sus usuarios.
		Los usuarios pueden entonces mantener su código actualizado y compartirlo online}
}

\newglossaryentry{stackoverflow}{
	name=StackOverflow,
	description={StackOverflow es un popular sitio sobre programación, en donde usuarios
		suelen realizar preguntas a problemas puntuales y otros usuarios las responden,
		transformando al sitio en una suerte de foro de intercambio de conocimientos sobre el
		tópico}
}

\newglossaryentry{mvnrepo}{
	name=MVNRepo,
	description={MVNRepo o Maven Repository, es un sitio web alojado en la dirección
		\emph{http://mvnrepository.com/} que permite buscar una amplia cantidad de paquetes
		\emph{jar} o \emph{war} que podemos importar en nuestro proyecto y nos provee del
		código necesario para agregarlo al mismo a través de distintos \depmgr}
}

\newglossaryentry{webfonts}{
	name=Fuentes web (Webfonts),
	description={Las fuentes web consisten en tipografías capas de ser utilizadas en el
		navegador sin necesidad de ser instaladas en el equipo del usuario. Previo a la
		creación de esta tecnología, solamente era posible utilizar un numero limitado de
		fuentes comunes a todos+los sistemas operativos}
}

\newglossaryentry{cdn}{
	name=CDN,
	description={ Una red de distribución de contenido consiste en una serie de servidores
		en \internet distribuidos en distintos puntos geográficos. Los usuarios solicitan
		contenido a estos servidores y es siempre el servidor más próximo a la ubicación
		del usuario el que entrega el mismo, disminuyendo los tiempos de descarga}
}

\newglossaryentry{node}{
	name=Node.js,
	description={Node.js es un entorno de programación en la capa del servidor basado en el
		lenguaje de programación \js, con I/O de datos en una arquitectura orientada a eventos
		y basado en el motor \js V8. Fue creado con el enfoque de ser útil en la creación de
		programas de red altamente escalables, como por ejemplo, servidores web}
}

\newglossaryentry{xml}{
	name=XML,
	description={XML es un lenguaje de marcas utilizado para almacenar datos en forma
		legible. XML se propone como un estándar para el intercambio de información
		estructurada entre diferentes plataformas. Se puede usar en bases de datos, editores
		de texto, hojas de cálculo y casi cualquier cosa imaginable}
}

\newglossaryentry{git}{
	name=Git,
	description={Git es un software de control de versiones distribuido, pensando en la
		eficiencia y la confiabilidad del mantenimiento de versiones de aplicaciones cuando
		estas tienen un gran número de archivos de código fuente}
}

\newglossaryentry{svn}{
	name=SVN,
	description={Subversion (SVN) es una herramienta de control de versiones open source
		basada en un repositorio cuyo funcionamiento se asemeja enormemente al de un sistema
		de ficheros}
}

\newglossaryentry{http}{
	name=HTTP,
	description={HTTP es el protocolo usado en cada transacción de Internet. HTTP define la
		sintaxis y la semántica que utilizan los elementos de software de la arquitectura web
		(clientes, servidores, proxies) para comunicarse. Es un protocolo orientado a
		transacciones y sigue el esquema petición-respuesta entre un cliente y un servidor}
}

\newglossaryentry{parser}{
	name=Parser,
	description={Un analizador sintáctico (o parser) es una de las partes de un compilador
		que transforma su entrada en un árbol de derivación}
}

\newglossaryentry{freesoft}{
	name=Software Libre,
	description={Software Libre es la denominación del software que respeta la libertad de
		todos los usuarios que adquirieron el producto y, por tanto, una vez obtenido el
		mismo puede ser usado, copiado, estudiado, modificado, y redistribuido libremente
		de varias formas}
}

\newglossaryentry{opensource}{
	name=Open Source,
	description={Código abierto es la expresión con la que se conoce al software distribuido
		y desarrollado libremente. Se focaliza más en los beneficios prácticos (acceso al
		código fuente) que en cuestiones éticas o de libertad que tanto se destacan en el
		software libre}
}

\newglossaryentry{jar}{
	name=JAR,
	description={El JAR es una extensión de archivo utilizada por aplicaciones de
		la plataforma \java. Es un archivo comprimido que contiene en su interior
		código \bytecode \java junto con metadata. Son, básicamente, programas que
		corren en la \jvm.}
}

\makeglossaries

\DeclareAcronym{pc}{
	short = PC,
	long = Computadora Personal,
	long-plural-form = Computadoras Personales,
	foreign = Personal Computer
}

\DeclareAcronym{darpa}{
	short = DARPA,
	long = Agencia de Proyectos de Investigación Avanzados de Defensa,
	foreign = Defense Advanced Research Projects Agency
}

\DeclareAcronym{html}{
	short = HTML,
	long = Lenguaje de Marcas de Hipertexto,
	foreign = HyperText Markup Language
}

\DeclareAcronym{ria}{
	short = RIA,
	long = Aplicacion de Internet Enriquecida,
	long-plural = Aplicaciones de Internet Enriquecidas,
	foreign = Rich Internet Application
}

\DeclareAcronym{css}{
	short = CSS,
	long = Hoja de Estilo en Cascada,
	long-plural = Hojas de Estilo en Cascadas,
	foreign = Cascade Style Sheet
}

\DeclareAcronym{html5}{
	short = HTML5,
	long = Lenguaje de Marcas de Hipertexto versión 5,
	foreign = HyperText Markup Language version 5
}

\DeclareAcronym{jvm}{
	short = JVM,
	long = Máquina Virtual de Java,
	long-plural = Máquinas Virtuales de Java,
	foreign = Java Virtual Machine
}

\DeclareAcronym{cdn}{
	short = CDN,
	long = Red de Distribución de Contenidos,
	long-plural = Redes de Distribución de Contenidos,
	foreign = Content Delivery Network
}

\DeclareAcronym{xml}{
	short = XML,
	long = Lenguaje de Marcado Extensible,
	foreign = eXtensible Markup Language
}

\DeclareAcronym{http}{
	short = HTTP,
	long = Lenguaje de Transferencia de Hipertexto,
	foreign = HyperText Transfer Language
}