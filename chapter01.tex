\section{Introducción}
\label{sec:intro}

Historia de los sistemas Java crecientes
Historia de los lenguajes para JVM
Frameworks WEB actuales
Problematica
Forma de agregar archivos a frameworks actuales
Manual
CDN
Interdependencias
Soluciones Maven e Ivy
Bower y Composer

En la presente sección intentaré explicar brevemente el contexto en el que se ha
desarrollado esta solución.\\
Comenzaré por enumerar los distintos factores que inciden sobre la elección del lenguaje
de programación a utilizar. Luego pasaremos a mostrar el problema en detalle. Finalmente
enumeraré soluciones existentes a problemas similares o al problema puntual, y mostraremos
en donde son útiles y donde no.

\subsection{Sobre los sistemas web}
\label{subsec:intro:about_web}

Si bien no es el objetivo de este documento explicar en detalle la historia de los cambios
en las arquitecturas de software, en este apartado intentaré dar una breve muestra de los
cambios que ha sufrido la industria de desarrollo de sistemas en los últimos tiempos, y
como estos han llevado a que los sistemas web sean hoy en día una de las opciones más
utilizadas, y más rentables.

\subsubsection{Estado anterior a los sistemas web}
\label{subsubsec:intro:about_web:previous_pc}

En las décadas de 1960 y 1970, comienza en el mundo un proceso lento pero incremental de
computarización de la información. Las computadoras dejan de ser \emph{"juguetes"}
cientificos para pasar a ser complejas maquinarias con verdadera utilidad practica
en distintas industrias.\\
En este contexto, varias empresas comienzan a \emph{informatizarse}. Bancos, empresas
petroleras y otros compran grandes y complejas computadoras capaces de almacenar toda
su información, o procesar complejos cálculos matemáticos en poco tiempo.\\
En este contexto, múltiples empresas como \emph{IBM}, \emph{RCA}, \emph{General Electric},
etc. comenzaron a fabricar gigantescas computadoras de precio cada vez más económico.
Estas maquinas evolucionarían poco a poco hasta convertirse en lo que hoy llamamos
\mainframes.\\
En este sentido, \citefullauthor{Stephens:2008:BOOK} en su libro
\citetitle{Stephens:2008:BOOK} \citeyearsq{Stephens:2008:BOOK} describe a un \mainframe como
una computadora muy grande, con una base de datos de alto rendimiento, a la cual se accede
desde una terminal remota, es decir, una maquina \emph{"tonta"}, sin mucha otra finalidad
más que la de conectarse al servidor central para realizar peticiones de datos y mostrarle
los resultados al usuario.\\
\citeauthor{Stephens:2008:BOOK} también destaca los beneficios de los \mainframes por sobre
una computadora personal a nivel empresarial. Las características que menciona son:
\begin{description}
	\item[Integridad de datos]: Los datos \textsc{deben} ser correctos.
	\item[Rendimiento]: Se debe procesar gran cantidad de datos.
	\item[Respuesta]: La respuesta debe ser inmediata.
	\item[Recuperación ante desastres]: En caso de un fallo, se debe volver a estar online
		inmediatamente.
	\item[Usabilidad]: Debe hacer lo que se requiere, cuando se requiere.
	\item[Confiabilidad]: No debe fallar y siempre debe estar disponible.
	\item[Auditoria]: Ser capaz de saber quien realizo que acción en el equipo.
	\item[Seguridad]: Solo aquellos que pueden realizar una acción pueden hacerlo.
\end{description}
Estas características son todavía buscadas en los grandes \mainframes de la actualidad,
y resultan fundamentales en emprendimientos críticos, por ejemplo, sistemas bancarios,
programas de control de acciones financieras, sistemas petroleros o el software que
controla misiones espaciales.\\
\citefullauthor{Elliot:2008:BOOK} ha descripto que las tendencias de la
tecnología utilizada, no obedecen solamente a motivaciones económicas, sino a un
complejo entramado de relaciones entre empresas que comparten una visión utópica sobre
la tecnología en cuestión \citesq[pag 3]{Elliot:2008:BOOK}. Así, las empresas
comercializadoras de \mainframes enarbolando como bandera principal los beneficios
planteados por \citeauthor{Stephens:2008:BOOK}, lograron generar una visión utópica en
las empresas usuarias para posicionar la tecnología dominante durante largos años.
 
\subsubsection{Aparición de las computadoras personales}
\label{subsubsec:intro:about_web:pc_era}

A partir de fines de los 70`s y comienzos de los 80`s, las computadoras personales
(\emph{PC}) comienzan a ser cada vez más accesibles \citesq[cap 4]{Allan:2001:BOOK}.
La liberación de ARPANET por parte de DARPA\footnote{
	DARPA acrónimo de la expresión en inglés \emph{\textbf{D}efense \textbf{A}dvanced
	\textbf{R}esearch \textbf{P}rojects \textbf{A}gency} (Agencia de Proyectos de
	Investigación Avanzados de Defensa) es una agencia del Departamento de Defensa de
	Estados Unidos responsable del desarrollo de nuevas tecnologías para uso militar.
	Fue creada en 1958 como consecuencia tecnológica de la llamada Guerra Fría.
} en 1983 da nacimiento a la Internet. Estos dos cambios suponen un quiebre
en las tecnologías de uso empresarial.\\
El \emph{PC}, sumado a la conectividad y nuevas tecnologías tanto en sistemas como en
lenguajes de programación llevaron a que los \mainframes comenzaran
a perder terreno.\\
La posibilidad de ejecutar programas en cada \emph{PC} dio lugar a nuevas arquitecturas
cliente-servidor. Ahora los usuarios corrian los programas en su propio equipo, permitiendo
aumentar la velocidad de respuesta. El programa se conectaba a Internet y sincronizaba
información con un servidor central solo cuando fuera necesario.\\
Esta arquitectura permitiría una más rápida evolución del software por sobre la opción de
tener el sistema en un \mainframe, como explican  \citefullauthor{Duncan:1996:ARTICLE} en su
artículo \citetitle{Duncan:1996:ARTICLE}. Sin embargo, como contrapartida  estos cambios
suponen un alto costo de mantenimiento, y en caso de poseer equipos con distintos sistemas
operativos, un gasto extra en el desarrollo. Esto se debe a la necesidad de generar un
cliente para cada sistema operativo, y la necesidad de actualizar los mismos en caso de
futuros cambios\footnote{
	Recuerde el lector que el software se distribuía en esa época en medios físicos, por
	lo que la instalación de una nueva versión del sistema requería presencia física en el
	equipo.
}.
demuestra que, independientemente de los costos, el factor más incidente en el cambio es la
necesidad de mantener el sistema al día con las necesidades de la empresa.

\subsubsection{Navegadores y primeros sistemas web}
\label{subsubsec:intro:about_web:web}

Tras la aparición de los navegadores web, como \emph{Netscape} e \emph{Internet Explorer},
a comienzos de 1990, se comenzaron a desarrollar los primeros sistemas web.\\
Los mismos, consistían básicamente en documentos dinámicos, generados con información
tomada de una base de datos. Gran influencia tuvieron en este tipo de soluciones, la
creación de lenguajes de programación y tecnologías pensadas para generar documentos
\emph{HTML}\footnote{
	HTML, siglas de \emph{HyperText Markup Language} (lenguaje de marcas de hipertexto), es
	un lenguaje de marcado para la elaboración de páginas web. Fué desarrollado por Tim
	Berners-Lee en 1991 y rápidamente se convirtió en un estándar.
} dinámicamente \citesq[pag 2]{Hunter:2001:BOOK}, como \emph{PHP}, \emph{ASP} y los
\emph{Servlets} de \emph{Java}.
La posibilidad de generar \emph{aplicaciones} que corran en el navegador, solucionó las
problemáticas asociadas a el mantenimiento de los sistemas y al desarrollo para múltiples
sistemas operativos. Sin embargo, esta vez, la naturaleza de HTML limitaba la capacidad
de las aplicaciones a sencillos formularios, botones y enlaces.\\
Estas limitaciones hicieron que no fuera sino hasta con la aparición de las primeras
tecnologías complementarias a HTML, que las aplicaciones web comenzarían a cobrar
relevancia, dando lugar a las \emph{Rich Internet Applications} (Aplicaciones de
Internet Enriquecidas).

\subsubsection{Rich Internet Applications}
\label{subsubsec:intro:about_web:rias}

Las \emph{Rich Internet Applications} (RIAs) surgen para compensar aquellas faltantes que
presentaban las aplicaciones web frente a las tradicionales de cliente-servidor de
escritorio.\\
En un primer lugar, complementos a HTML en forma de \emph{plugins} para los navegadores
fueron creados. Uno de los más populares fue \emph{Flash} de \emph{Adobe}, que todavía
continua presente hoy en día. Java tampoco se quedo atrás y presentó sus \emph{Servlets},
una forma de embeber código Java en páginas HTML.\\
Por primera vez se podían emular complejos comportamientos y componentes mediante estilo
y comportamiento en el navegador del usuario. Desde arrastras y soltar, ordenar elementos
hasta crear complejos gráficos.\\
Estas aplicaciones comenzaron a copar los mercados empresariales, ya que permitían ahorrar
en mantenimiento y desarrollo, a la vez que brindaban la alta usabilidad que se demandaba
de un sistema empresarial.\\
Cada vez más soluciones comenzaron a surgir, \emph{Adobe Flex}, \emph{Microsoft Silverlight}
y \emph{JavaFX} \citesq{Clarke:2009:BOOK}. Sin embargo, la falta de estándares llevaría a que
se desarrollaran con fuerza los estándares \emph{CSS}\footnote{
	CSS, acronimo de \emph{\textsc{C}ascading \textsc{S}tyle \textsc{S}heets} (Hojas de
	Estilo en Cascada) es un lenguaje utilizado para describir el aspecto y el formato de un
	documento escrito en lenguaje HTML o similar.
} y \emph{JavaScript}\footnote{
	JavaScript (abreviado comúnmente JS) es un lenguaje de programación interpretado,
	utilizado principalmente del lado del cliente, ya que es implementado como parte de un
	navegador web, permitiendo mejoras en la interfaz de usuario y páginas web dinámicas.
}, tecnologías libres y disponibles en cualquier navegador sin necesidad de instalar
software adicional al equipo, caso que si ocurría con los \emph{plugins} antes mencionados.\\
Así, en los últimos años,y gracias también al auge de los smartphones y tablets, incapaces
de correr los anteriormente mencionados plugins, el llamado \emph{HTML5}\footnote{
	HTML5 (\emph{\textsc{H}yperText \textsc{M}arkup \textsc{L}anguage}, versión 5) es la
	quinta revisión importante del lenguaje básico de marcado de documentos para Internet.
	Su nombre suele hacer referencia no solo al nuevo estándar, todavía experimental de este
	lenguaje, sino también a las tecnologías que lo acompañan, CSS en su versión 3, y
	JavaScript.
} pasaría a transformarse en el estándar para el desarrollo de RIAs \citesq{David:2013:BOOK}.\\
La tendencia marca que las RIAs desarrolladas bajo el estándar HTML5 seguirán siendo la
norma durante los años venideros. Multples frameworks han aparecido capaces de generar
rápidamente sistemas web con un \emph{front-end} completamente realizado en HTML5.\\



\subsection{Acerca de la Java Virtual Machine (JVM)}
\label{subsec:intro:about_jvm}

La maquina virtual de Java (\emph{Java Virtual Machine}, o simplemente \emph{JVM}) es una
parte fundamental de la plataforma del lenguaje Java, creada originalmente por \emph{Sun
Microsystems}. La maquina virtual crea una capa de abstracción entre el sistema operativo
y el código Java compilado (\emph{Bytecode}). De esta forma, el código Java puede ser
compilado a bytecode una sola vez, y correrse en cualquier sistema que cuente con una
JVM.\\
Con la popularidad de Java, comenzaron a surgir una serie de lenguajes que compilan a este
bytecode. Estos lenguajes se presentan como alternativas para los programadores de
la plataforma que buscan una alternativa "superior", en algún aspecto, a Java. Groovy por
ejemplo, se presenta como un lenguaje dinámico, similar a Ruby; Clojure es una extensión
de Lisp apuntado a la Programación Concurrente; Scala es un lenguaje que presenta
características tanto de Programación Funcional como de Programación Orientada a Objectos.
También existen otra numerosa cantidad de lenguajes, algunos experimentales, algunos con
una base creciente de usuarios, que corren sobre la plataforma, como también una amplia
cantidad de adaptaciones de lenguajes populares para que corran en la JVM, como JRuby y
Jython, versiones de Ruby y Python que compilan a bytecode java.\\
Una característica interesante de estos lenguajes es que el codigo compilado suele ser
compatible entre si, es decir, código escrito en Java puede ser ejecutado en Scala,
codigo Groovy puede ser usado desde Clojure, etc. (Con algunas excepciones).\\



\subsection{Crecimiento de la JVM como plataforma}
\label{subsec:intro:jvm_growth}

Java es, casi sin cuestionamientos \footnote{
	Se puede argumentar que la medición de la popularidad de
	los lenguajes de programación es una ciencia inexacta, pues los factores de medición no
	se encuentran determinados. Sumado a esto, la mayoría de los resultados se basa en
	cantidad de lineas de código y menciones online. Esto no tiene en consideración las
	diferencias entre un lenguaje y otro, ya que algunos son necesariamente más
	verborragicos que otros (Por ejemplo, dos códigos que realizan exactamente la misma
	acción en dos lenguajes distintos pueden resultar en cantidades significativamente
	distintas de lineas de código entre uno y otro, solo por la sintaxis y la naturaleza
	misma del lenguaje.) Así, cada estadista toma distintas variables, le otorga distinto
	peso a las mismas y obtendrá distintos resultados.
}, uno de los lenguajes más populares. Así lo demuestran las estadísticas de análisis de
popularidad de lenguajes, como la renombrada estadística TIOBE que a la fecha de esta
publicación ubica a Java en segundo lugar \citesq{TIOBE:2014:ONLINE}. La estadística
\emph{"transparente"} creada por \citefullauthor{Montmollin:2013:ONLINE}, que publica el
código de forma open source y sus reglas online, muestra al lenguaje de Oracle en la misma
posición \citesq{Montmollin:2013:ONLINE}, mientras que la otra herramienta open source que
permite medir la popularidad de los lenguajes, PyPL, de
\citefullauthor{Carbonelle:2014:ONLINE}, lo muestra en primer lugar
\citesq{Carbonelle:2014:ONLINE}.\\
Además, \citefullauthor{Kunst:2014:ONLINE} ha desarrollado un gráfico en donde muestra la
cantidad de líneas de código en GitHub\footnote{
	GitHub es un popular sitio web que permite almacenar código a sus usuarios. Los usuarios
	pueden entonces mantener su código actualizado y compartirlo online.
} por sobre las menciones en StackOverflow\footnote{
StackOverflow es un popular sitio de preguntas y respuestas sobre programación.
}, mostrando a Java por sobre los más populares, pero también a los lenguajes que corren
sobre la Java Virtual Machine (JVM)\footnote{
	Entre los que se pueden observar destacan Scala, Clojure y Groovy. Aunque no
	mencionados, son muy populares también las versiones para JVM de Ruby, Python y
	JavaScript, JRuby, Jython y Rhino. También existen otros lenguajes, con menor
	popularidad, pero dignos de mención como AspectJ, Ceylon, Fantom, Fortress, Frege, Gosu,
	Ioke, Jelly, Kotlin, Mirah, Processing, X10 y Xtend.
} como de alto interés.\\
Queda entonces patente que los lenguajes que se ejecutan en la JVM tienen un uso cada vez
mayor. Empero, la diferencia se volvería aun más grande si consideramos el uso de los
lenguajes para desarrollo de soluciones web.\\
Si bien no existen estadísticas que confirmen esto, es prácticamente reconocido por
cualquier programador que la ventaja que se lee en las estadísticas por parte del lenguaje
C, es producto del desarrollo, no de sistemas web, sino de pequeñas herramientas de
escritorio. Asimismo, la popularidad creciente de Objective-C, se debe al uso del mismo en
las aplicaciones móviles de iPhone y iPad. Finalmente, el amplio desarrollo que presenta en
las encuestas JavaScript, no compite necesariamente con los sistemas hechos en JVM sino que
de hecho, tiende a complementarlos, ya que es una parte importante de las
RIAs\ref{subsubsec:intro:about_web:rias}.\\

\subsection{Desarrollo de sistemas web en la JVM}
\label{subsec:intro:jvm_dev}

\subsubsection{Estructura común de sistemas web}
\label{susubbsec:intro:jvm_dev:structure}

\subsubsection{Reutilización de código mediante dependencias}
\label{susubbsec:intro:jvm_dev:dependencies}

\subsubsection{Dependencias manejadas de backend}
\label{susubbsec:intro:jvm_dev:backend_dependencies}

\subsubsection{Dependencias no manejadas de front-end}
\label{susubbsec:intro:jvm_dev:frontend_dependencies}

\subsubsection{CDN}
\label{susubbsec:intro:jvm_dev:cdns}

\subsubsection{Dependencias manejadas de front-end}
\label{susubbsec:intro:jvm_dev:frontend_managed}

