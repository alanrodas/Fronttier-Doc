\section{Introducción}

Historia de los sistemas Java crecientes
Historia de los lenguajes para JVM
Frameworks WEB actuales
Problematica
Forma de agregar archivos a frameworks actuales
Manual
CDN
Interdependencias
Soluciones Maven e Ivy
Bower y Composer

En la presente sección intentaré explicar brevemente el contexto en el que se ha
desarrollado esta solución.\\
Comenzaré por enumerar los distintos factores que inciden sobre la elección del lenguaje
de programación a utilizar. Luego pasaremos a mostrar el problema en detalle. Finalmente
enumeraré soluciones existentes a problemas similares o al problema puntual, y mostraremos
en donde son útiles y donde no.

\subsection{Sobre los sistemas web}

Si bien no es el objetivo de este documento explicar en detalle la historia de los cambios
en las arquitecturas de software, en este apartado se intentará explicar los cambios que ha
sufrido la industria de desarrollo de software en los últimos tiempos, y como estos han
llevado a que los sistemas web sean hoy en día una de las opciones más utilizadas, y más
rentables en cuanto a desarrollo de software se refiere. Además, se intentará explicar en
forma sencilla la estructura básica de este tipo de soluciones informáticas.

\subsubsection{Estado anterior a los sistemas web}
En las décadas de 1960 y 1970, comienza en el mundo un proceso lento pero incremental de
computarización de la información. Las computadoras dejan de ser \emph{"juguetes"} para
pasar a ser complejas maquinarias con verdadera utilidad practica.\\
En este contexto, varias empresas comienzan a \emph{informatizarse}. Bancos y otros compran
grandes y complejas computadoras capaces de almacenar toda su información, procesarla y
presentar resultados en poco tiempo.\\
Con este contexto inicial, los sistemas informáticos empresariales evolucionaron a con una
arquitectura cliente-servidor, con tecnologías que hoy son referidas como \mainframes.\\
En este sentido, \citefullauthor{Stephens:2008:BOOK} en su libro
\citetitle{Stephens:2008:BOOK} \citeyearsq{Stephens:2008:BOOK} describe a un \mainframe como
una computadora muy grande, con una base de datos de alto rendimiento, a la cual se accede
desde una terminal remota (o un emulador de la misma), es decir, una maquina
\emph{"tonta"}, sin mucha otra finalidad más que la de conectarse al servidor central para
realizar peticiones de datos y mostrarle los resultados al usuario.\\
\citeauthor{Stephens:2008:BOOK} también destaca los beneficios de los \mainframes por sobre
una computadora personal a nivel empresarial. Las características que menciona son:
\begin{description}
	\item[Integridad de datos]: Los datos \textsc{deben} ser correctos.
	\item[Rendimiento]: Se debe procesar gran cantidad de datos.
	\item[Respuesta]: La respuesta debe ser inmediata.
	\item[Recuperación ante desastres]: En caso de un fallo, se debe volver a estar online
		inmediatamente.
	\item[Usabilidad]: Debe hacer lo que se requiere, cuando se requiere.
	\item[Confiabilidad]: No debe fallar y siempre debe estar disponible.
	\item[Auditoria]: Ser capaz de saber quien realizo que acción en el equipo.
	\item[Seguridad]: Solo aquellos que pueden realizar una acción pueden hacerlo.
\end{description}
Estas características resultan fundamentales en emprendimientos críticos, por ejemplo,
sistemas bancarios, o programas de control de acciones financieras.\\
 \citefullauthor{Elliot:2008:BOOK} ha descripto que las tendencias de la
 tecnología utilizada, no obedecen solamente a motivaciones económicas, sino a un
 complejo entramado de relaciones entre empresas que comparten una visión utopica sobre
 la tecnología en cuestión \citesq{pag 3, Elliot:2008:BOOK}. Así, los \mainframes se
 lograron posicionar como la tecnología dominante gracias a la visión utópica generada
 en las empresas por los vendedores de estos costosos sistemas (principalmente IBM),
 enarbolando como bandera principal los beneficios planteados por
 \citeauthor{Stephens:2008:BOOK}.
 
\subsubsection{Aparición de las computadoras personales}

A partir de fines de la década de 1970 y comienzos de la década de 1980, las computadoras
personales (\emph{PC}) comienzan a ser cada vez más accesibles. La liberación de ARPANET
por parte de DARPA\footnote{
	DARPA acrónimo de la expresión en inglés \emph{\textbf{D}efense \textbf{A}dvanced
	\textbf{R}esearch \textbf{P}rojects \textbf{A}gency} (Agencia de Proyectos de
	Investigación Avanzados de Defensa) es una agencia del Departamento de Defensa de
	Estados Unidos responsable del desarrollo de nuevas tecnologías para uso militar.
	Fue creada en 1958 como consecuencia tecnológica de la llamada Guerra Fría.
} en 1983 da nacimiento a la Internet. Estos dos cambios suponen un quiebre
en las tecnologías de uso empresarial.\\
El desarrollo de nuevos lenguajes de programación, computadoras potentes de bajo costo, y
la conectividad, pero por sobre todo el \emph{PC} llevaron a que los \mainframes comenzaran
a perder terreno. De esta forma, la industria comienza a torcer lentamente hacia el desarrollo
de aplicaciones cliente servidor que ya no radican solamente en terminales \emph{"tontas"},
sino que los programas comienzan a correr en las computadoras de los usuarios, procesar datos
en ellas y luego comunicarse con un servidor para sincronizar la información en caso de ser
necesario.\\
Debido a la naturaleza misma de la tecnología empleada en el desarrollo de estas soluciones,
estos cambios suponen un alto costo de mantenimiento y de producción. los dos factores más
incidentes, son la necesidad de generar un cliente para cada sistema operativo, y la necesidad
de actualizar los mismos en caso de futuros cambios\footnote{
	Recuerde el lector que el software se distribuía en esa época en medios físicos, por
	lo que la instalación de una nueva versión del sistema requería presencia física en el
	equipo.
}.

\subsubsection{Navegadores y primeros sistemas web}

Tras la aparición de los navegadores web, como \emph{Netscape} e \emph{Internet Explorer},
a comienzos de 1990, se comenzaron a desarrollar los primeros sistemas web.\\
Los mismos, consistían básicamente en documentos dinámicos, generados con información
tomada de una base de datos. Gran influencia tuvieron en este tipo de soluciones, la creación
de lenguajes de programación como \emph{PHP} o \emph{ASP} que trabajaban sobre el protocolo
de Internet para compartir \emph{HTML}\footnote{
	HTML, siglas de \emph{HyperText Markup Language} (lenguaje de marcas de hipertexto), es
	un lenguaje de marcado para la elaboración de páginas web. Fué desarrollado por Tim
	Berners-Lee en 1991 y rápidamente se convirtió en un estándar.
} generado dinámicamente.\\
La posibilidad de generar \emph{aplicaciones} que corran ene el navegador, solucionó las
problemáticas asociadas a el mantenimiento de los sistemas y al desarrollo para múltiples
sistemas operativos. Sin embargo, esta vez, la naturaleza de HTML limitaba la capacidad
de las aplicaciones a sencillos formularios, botones y enlaces.\\
Estas limitaciones hicieron que el modelo de sistemas web no fuese completamente adoptado.
No fue sino hasta con la aparición de las primeras tecnologías complementarias a HTML, que
las aplicaciones web comenzarían a cobrar más relevancia, dando lugar a las \emph{Rich
Internet Applications} (Aplicaciones de Internet Enriquecidas).

\subsubsection{Rich Internet Applications}

Las \emph{Rich Internet Applications} (RIAs) surgen para compensar aquellas faltantes que
presentaban las aplicaciones web fente a las tradicionales aplicaciones de escritorio
cliente-servidor.\\
En un primer lugar, complementos a HTML en forma de \emph{plugins} para los navegadores
fueron creados. Uno de los más populares fue \emph{Flash} de \emph{Adobe}, que todavía
continua presente hoy en día. Java tampoco se quedo atrás y presento sus \emph{servlets},
una forma de embeber código Java en páginas HTML.\\
Por primera vez, se podían emular complejos comportamientos y componentes mediante
reglas de estilo y comportamiento en el navegador del usuario. Desde arrastras y soltar,
ordenar elementos hasta crear complejos gráficos.\\
Estas aplicaciones comenzaron a copar los mercados empresariales, ya que permite a ahorrar
en mantenimiento y desarrollo, a la vez que permitía una alta usabilidad. Con la aceleración
de las conexiones a Internet y el incremento de dispositivos para conectarse a la misma
(tablets, teléfonos inteligentes, etc.) estas soluciones posibilitaban además beneficios
como la posibilidad de en equipos distribuidos, o acceder a servicios de forma remota.\\
Cada vez más soluciones comenzaron a surgir, \emph{Adobe Flex}, \emph{Microsoft Silverlight},
etc. Sin embargo, la falta de estándares llevaría a que se desarrollaran con fuerza los
lenguajes y complementos \emph{CSS}\footnote{
	CSS, acronimo de \emph{\textsc{C}ascading \textsc{S}tyle \textsc{S}heets} (Hojas de
	Estilo en Cascada) es un lenguaje utilizado para describir el aspecto y el formato de un
	documento escrito en lenguaje HTML o similar.
} y \emph{JavaScript}\footnote{
	JavaScript (abreviado comúnmente JS) es un lenguaje de programación interpretado,
	utilizado principalmente del lado del cliente, ya que es implementado como parte de un
	navegador web, permitiendo mejoras en la interfaz de usuario y páginas web dinámicas.
}, tecnologías libres y disponibles en cualquier navegador sin necesidad de instalar
software adicional en el equipo, como si ocurría con los \emph{plugins}.\\
Así, en los últimos años, el llamado \emph{HTML5}\footnote{
	HTML5 (\emph{\textsc{H}yperText \textsc{M}arkup \textsc{L}anguage}, versión 5) es la quinta revisión importante del
	lenguaje básico de marcado de documentos para Internet. Su nombre suele hacer
	referencia no solo al nuevo estándar, todavía experimental de este lenguaje, sino
	también a las tecnologías que lo acompañan, CSS en su versión 3, y JavaScript.
} pasaría a transformarse en el estándar para el desarrollo de RIAs. Estas tecnologías
se ha vuelto cada vez más comunes. La tendencia marca que las RIAs desarrolladas bajo el
estándar HTML5 seguirán siendo la tendencia durante los años venideros.\\

\subsection{Crecimiento de la JVM como plataforma}

Java es, casi sin cuestionamientos \footnote{
	Se puede argumentar que la medición de la popularidad de
	los lenguajes de programación es una ciencia inexacta, pues los factores de medición no
	se encuentran determinados. Sumado a esto, la mayoría de los resultados se basa en
	cantidad de lineas de código y menciones online. Esto no tiene en consideración las
	diferencias entre un lenguaje y otro, ya que algunos son necesariamente más
	verborragicos que otros (Por ejemplo, dos códigos que realizan exactamente la misma
	acción en dos lenguajes distintos pueden resultar en cantidades significativamente
	distintas de lineas de código entre uno y otro, solo por la sintaxis y la naturaleza
	misma del lenguaje.) Así, cada estadista toma distintas variables, le otorga distinto
	peso a las mismas y obtendrá distintos resultados.
}, uno de los lenguajes más populares. Así lo demuestran las estadísticas de análisis de
popularidad de lenguaje, como la popular estadística TIOBE que a la fecha de esta
publicación ubica a Java en segundo lugar \citesq{TIOBE:2014:ONLINE}. La estadística
\emph{"transparente"} creada por \citefullauthor{Montmollin:2013:ONLINE}, que publica el
código de forma open source y sus reglas online, muestra al lenguaje de Oracle en la misma
posición \citesq{Montmollin:2013:ONLINE}, mientras que la otra herramienta open source que
permite medir la popularidad de los lenguajes, PyPL, de
\citefullauthor{Carbonelle:2014:ONLINE}, muestra lo muestra en primer lugar
\citesq{Carbonelle:2014:ONLINE}.\\
Además, \citefullauthor{Kunst:2014:ONLINE} ha desarrollado un gráfico en donde muestra la
cantidad de líneas de código en GitHub\footnote{
	GitHub es un popular sitio web que permite almacenar código a sus usuarios. Los usuarios
	pueden entonces mantener su código actualizado y compartirlo online.
} por sobre las menciones en StackOverflow\footnote{
StackOverflow es un popular sitio de preguntas y respuestas sobre programación.
}, mostrando a Java por sobre los más populares, pero también a los lenguajes que corren
sobre la Java Virtual Machine (JVM)\footnote{
	Entre los que se pueden observar destacan Scala, Clojure y Groovy. Aunque no
	mencionados, son muy populares también las versiones para JVM de Ruby, Python y
	JavaScript, JRuby, Jython y Rhino. También existen otros lenguajes, con menor
	popularidad, pero dignos de mención como AspectJ, Ceylon, Fantom, Fortress, Frege, Gosu,
	Ioke, Jelly, Kotlin, Mirah, Processing, X10 y Xtend.
} como de alto interés.\\
Queda entonces patente que los lenguajes que se ejecutan en la JVM tienen un uso cada vez
más creciente. Empero, la diferencia se volvería aun mayor si consideramos el uso de los
lenguajes para desarrollo de soluciones empresariales web.\\
Si bien no existen estadísticas que confirmen esto, es prácticamente reconocido por
cualquier programador que la ventaja que se lee en las estadísticas por parte del lenguaje
C, son producto del desarrollo, no de sistemas web, sino de pequeñas herramientas de
escritorio. Asimismo, la popularidad creciente de Objective-C, se debe al uso del mismo en
las aplicaciones móviles de iPhone. Finalmente, el amplio desarrollo que presenta en las
encuestas JavaScript, no compite necesariamente con los sistemas hechos en JVM sino que de
hecho, tiende a complementarlos.\\

\subsection{Desarrollo de sistemas web en la JVM}
\subsubsection{Estructura común de sistemas web}
\subsubsection{Reutilización de código mediante dependencias}
\subsubsection{Dependencias manejadas de backend}
\subsubsection{Dependencias no manejadas de front-end}
\subsubsection{CDN}
\subsubsection{Dependencias manejadas de front-end}