\section{Solución Propuesta}

De analizar el apartado anterior se puede percibir que sería de amplio
interés para los desarrolladores contar con una herramienta que corra sobre
la \jvm y que permita manejar las \dependencies de la \viewtier.\\
Este trabajo propone la generación de una solución a este problema desarrollada
en \scala y usando \sbt para manejar las \dependencies en el mismo.\\

\subsection{Características deseadas}

Se tomarán como requerimientos una serie de características que a los efectos
parecen deseables en una herramienta que intenta posicionarse como el
estándar para manejo de \dependencies de la \viewtier en lenguajes que corren
sobre la plataforma.\
En primer lugar, deberá correr enteramente sobre la \jvm, evitando depender de
la instalación de software adicional en el equipo y haciendolo independiente
del sistema operativo, por lo cual, es necesario que se encuentre desarrollada
en un lenguaje que compile a \bytecode \java.
Adicionalmente deberá poder ser utilizada desde cualquier otro lenguaje de
la \jvm de forma natural, es decir, que en caso de características incompatibles
de los lenguajes, debería proveer una capa de abstracción de forma tal que
el uso sea transparente y completo. Incluso es deseable que pueda ser
ejecutada en forma de programa independiente desde la linea de comandos.\\
La herramienta deberá ser capaz de descargar y acomodar los paquetes que el
usuario determine como \dependencies en un archivo de configuración. Como se ha
mencionado en la sección \ref{susubbsec:intro:jvm_dev:view_managed}, distintas
herramientas utilizan distintos tipos de archivos de configuración. En lugar de
forzar un formato especifico para el mismo, el \depmgr desarrollado deberá
brindar al usuario la posibilidad de usar un formato para el mismo que se
adecue a sus necesidades\footnote{
	Por ejemplo, los usuarios de \emph{apache Maven} podrían optar por un archivo
	de configuración en \xml similar al que acostumbran y conocen, mientras
	que los usuarios de \sbt podrían elegir un archivo de configuración cuya
	sintaxis asemeje los usados en esa herramienta. Debería ser lo bastante flexible
	para permitir la generación de formatos de archivo de configuración por los
	mismos usuarios de la herramienta de forma sencilla.
}.\\
Al existir soluciones como \emph{Bower} o \emph{Component}\footnote{
	Estas soluciones no cumplen todos los requerimientos deseados, pero si son
	utilizadas con éxito en proyectos de desarrollo.
}, la herramienta debería complementarlas, permitiendo por ejemplo
utilizar sus archivos de configuración y sus repositorios.\\
Finalmente, sería ideal contar con un repositorio propio, en donde desarrolladores
puedan subir sus paquetes, o al menos registrar los mismos, permitiendo así
consultar la existencia de dependencias. Además, los usuarios podrían querer
generar sus propios repositorios y hacer que la herramienta consulte en
ellos\footnote{
	Muchas empresas de software suelen tener sus propios repositorios para
	garantizar disponibilidad de sus requerimientos o porque contienen código
	propietario usado internamente.
}. Algunos de estos repositorios podrían ser privados, y por tanto, se requeriría
acceso mediante un usuario y contraseña que la herramienta debe enviar de forma
correcta y segura.\\
Finalmente, podemos destacar el hecho de que, actualmente, muchas \dependencies
de la \viewtier consisten en múltiples archivos. La mayoría de los \depmgrs actuales
se basan en archivos, haciendo que se requiera una declaración en el archivo de
configuración por cada archivo de la dependencia, algo que no solo es poco práctico,
sino propenso a errores ya que el olvido de una declaración invalida completamente
la dependencia.\\
Finalmente, al existir numerosos \frameworks para distintos lenguajes de la \jvm
que apuntan al desarrollo de aplicaciones web, sería deseable la integración con los
mismos\footnote{
	Por ejemplo, múltiples \frameworks utilizan sistemas de templates para generar el
	código \html que finalmente será utilizado. Sería ideal poder brindar algún
	componente que integre los elementos descargados a los templates del usuario
	de forma automática.
}.\\
En forma de resumen y para mayor claridad, se enumeran a continuación todas las
características mencionadas:
\begin{itemize}
	\item Hecho enteramente sobre la \jvm
	\item Posibilidad de ser usado desde cualquier lenguaje de la \jvm
	\item Archivos de configuración personalizados, con varios formatos por defecto
	\item Múltiples repositorios
	\item Seguridad mediante autenticación en los repositorios
	\item Integración con estándares existentes de \depmgrs
	\item Integración con tecnologías web existentes en la \jvm
\end{itemize}
Otros requerimientos podrían surgir a futuro, pero no son tenidos en cuenta al
momento de desarrollar esta solución.

\subsection{Elección de tecnologías a utilizar}

Se ha determinado que la tecnología a usar será \scala. La elección se basa en la
facilidad que presenta este lenguaje para generar código compatible con otros de la
\jvm y la gran cantidad de características que presenta, las cuales permiten generar
gran funcionalidad en pocas lineas de código. Además, \scala soporta características
de análisis sintáctico (\emph{parsing}) sencillas de usar desde el lenguaje, que
resultan muy útiles cuando se requiere leer archivos de configuración. \sbt permite
automatizar las tareas de manejo de dependencias de forma rápida y sencilla, y es el
\depmgr de facto en \scala.\\
En algunos casos, tales como en el desarrollo en herramientas de integración con
tecnologías existentes, es posible que sea necesario el uso de otros lenguajes,
como \java, \clojure o \groovy.

\subsection{Alcance del presente trabajo}

El presente trabajo no pretende ser la solución completa a los requerimientos
planteados, los cuales son muy amplios y llevarían demasiado tiempo para abordar
por una sola persona. En cambio se enfocará en las características más importantes,
dejando asentadas las bases para el desarrollo de las partes no concluidas.
Así el enfoque será puesto en la estructura del sistema, orientada siempre a
la extensibilidad.\\
Se presenta como objetivo el desarrollo de al menos las 
siguientes características de forma completamente funcionales:
\begin{itemize}
	\item Desarrollo del núcleo del sistema
	\item Descargar dependencias desde la web vía HTTP
	\item Descargar desde \gls{github} mediante \git
	\item Descargar desde un repositorio \svn
	\item Lectura de archivos de configuración en \xml estilo Maven
	\item Lectura de archivos de configuración en \xml estilo Ivy
	\item Lectura de archivos de configuración en estilo \sbt
	\item Agregado de múltiples repositorios
	\item Usar la herramienta desde la linea de comandos
	\item Usar la herramienta desde \scala
	\item Usar la herramienta desde \java
	\item Integración de la herramienta con Maven
	\item Integración de la herramienta con \sbt
	\item Integración de la herramienta con Play! Framework
\end{itemize}
Si bien quedarán pendientes muchas tareas para realizar, estas serán abordadas
en futuros trabajos. Además, la herramienta será liberada como software libre con
licencia \emph{Apache versión 2}. Esto permitirá que la comunidad de desarrolladores
expanda la herramienta para funcionar con todos los lenguajes de la \jvm, en muchos
más \frameworks, y que se integre con una mayor cantidad de procesos.