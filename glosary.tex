\newglossaryentry{html}{
	name=HTML,
	description={HTML, siglas de \emph{HyperText Markup Language} (lenguaje de marcas de hipertexto), es
		un lenguaje de marcado para la elaboración de páginas web. Fué desarrollado por Tim
		Berners-Lee en 1991 y rápidamente se convirtió en un estándar, p}
}

\newglossaryentry{ria}{
	name=RIA,
	description={Una aplicación de Internet enriquecida es una aplicación que corre en el
		navegador del usuario y que permite emular complejos comportamientos y funcionalidades
		que antes solo eran disponible en sistemas de escritorio, tales como arrastras y soltar,
		ordenar elementos o crear complejos gráficos. Para esto hacen uso de distintas tecnologías,
		que pueden ser, programas que corren sobre el navegador en forma de complementos, o 
		aplicaciones creadas enteramente en HTML, CSS y JavaScript, p}
}

\newglossaryentry{css}{
	name=CSS,
	description={CSS es un lenguaje utilizado para describir el aspecto y el formato de un
		documento escrito en lenguaje HTML o similar. Actualmente es un estandar web y
		es soportado por prácticamente todos los navegadores, p}
}

\newglossaryentry{js}{
	name=JavaScript,
	description={JavaScript (abreviado comúnmente JS) es un lenguaje de programación interpretado,
		orientado a objetos y basado en prototipos, utilizado principalmente del lado del cliente.
		Este lenguaje es un estándar web ya que es implementado como parte de prácticamente todos
		los navegadores web, permitiendo interactividad del usuario con la interfaz sin necesidad
		de comunicarse constantemente con el servidor, p}
}

\newglossaryentry{html5}{
	name=HTML5,
	description={Se conoce como HTML5 a la
		quinta revisión importante del lenguaje básico de marcado de documentos para \internet.
		Su nombre suele hacer referencia no solo al nuevo estándar, todavía experimental de este
		lenguaje, sino también a las tecnologías que lo acompañan, CSS en su versión 3, y
		JavaScript, p}
}

\newglossaryentry{jvm}{
	name=JVM,
	description={La JVM es una parte fundamental de la plataforma del lenguaje Java,
		creada originalmente por \emph{Sun Microsystems}. La maquina virtual crea una capa de abstracción
		entre el sistema operativo y el código Java compilado (\emph{Bytecode}).
		De esta forma, el código Java puede ser compilado a \bytecode una sola vez, y correrse
		en cualquier sistema que cuente con una JVM, p}
}

\newglossaryentry{github}{
	name=GitHub,
	description={GitHub es un popular sitio web que permite almacenar código a sus usuarios. Los usuarios
		pueden entonces mantener su código actualizado y compartirlo online, p}
}

\newglossaryentry{stackoverflow}{
	name=StackOverflow,
	description={StackOverflow es un popular sitio sobre programación, en donde usuarios suelen realizar
		preguntas a problemas puntuales y otros usuarios las responden, transformando al sitio en una
		suerte de foro de intercambio de conocimientos sobre el tópico, p}
}

\newglossaryentry{mvnrepo}{
	name=MVNRepo,
	description={MVNRepo o Maven Repository, es un sitio web alojado en la dirección
		\emph{http://mvnrepository.com/} que permite buscar una amplia
		cantidad de paquetes \emph{jar} o \emph{war} que podemos importar en nuestro
		proyecto y nos provee del código necesario para agregarlo al mismo a través de
		distintos \depmgr, p}
}

\newglossaryentry{webfonts}{
	name=Fuentes web (Webfonts),
	description={Las fuentes web consisten en tipografías capas de ser utilizadas en el navegador sin
		necesidad de ser instaladas en el equipo del usuario. Previo a la creación de esta
		tecnología, solamente era posible utilizar un numero limitado de fuentes comunes a todos+los sistemas operativos, p}
}

\newglossaryentry{cdn}{
	name=CDN,
	description={ Una red de distribución de contenido consiste en una serie de servidores
		en \internet distribuidos en distintos puntos geográficos. Los usuarios solicitan
		contenido a estos servidores y es siempre el servidor más próximo a la ubicación
		del usuario el que entrega el mismo, disminuyendo los tiempos de descarga, p}
}

\newglossaryentry{node}{
	name=Node.js,
	description={Node.js es un entorno de programación en la capa del servidor basado en el lenguaje de programación \js,
		con I/O de datos en una arquitectura orientada a eventos y basado en el motor \js V8. Fue creado con el enfoque
		de ser útil en la creación de programas de red altamente escalables, como por ejemplo, servidores web, p}
}

\makeglossaries

\DeclareAcronym{pc}{
	short = PC,
	long = Computadora Personal,
	long-plural-form = Computadoras Personales,
	foreign = Personal Computer
}

\DeclareAcronym{darpa}{
	short = DARPA,
	long = Agencia de Proyectos de Investigación Avanzados de Defensa,
	foreign = Defense Advanced Research Projects Agency
}

\DeclareAcronym{html}{
	short = HTML,
	long = Lenguaje de Marcas de Hipertexto,
	foreign = HyperText Markup Language
}

\DeclareAcronym{ria}{
	short = RIA,
	long = Aplicacion de Internet Enriquecida,
	long-plural = Aplicaciones de Internet Enriquecidas,
	foreign = Rich Internet Application
}

\DeclareAcronym{css}{
	short = CSS,
	long = Hoja de Estilo en Cascada,
	long-plural = Hojas de Estilo en Cascadas,
	foreign = Cascade Style Sheet
}

\DeclareAcronym{html5}{
	short = HTML5,
	long = Lenguaje de Marcas de Hipertexto versión 5,
	foreign = HyperText Markup Language version 5
}

\DeclareAcronym{jvm}{
	short = JVM,
	long = Maquina Virtual de Java,
	long-plural = Maquinas Virtuales de Java,
	foreign = Java Virtual Machine
}

\DeclareAcronym{cdn}{
	short = CDN,
	long = Red de Distribución de Contenidos,
	long-plural = Redes de Distribución de Contenidos,
	foreign = Content Delivery Network
}