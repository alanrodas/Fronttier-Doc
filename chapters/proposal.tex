\section{Solución Propuesta: Manejador de \dependencies web para la JVM}
\label{sec:solution}

Basándonos en el problema de manejar \dependencies en la \viewtier presentado 
en el capítulo \ref{sec:intro}, es necesaria la implementación de una 
herramienta para los desarrolladores que corra sobre la \jvm y permita manejar 
las mencionadas \dependencies.\\
El presente trabajo propone como solución el desarrollo de un \depmgr 
implementado en 
\scala llamado \fronttier. El mismo utilizará \conffiles con distintos formatos 
para configurar las \dependencies requeridas. Adicionalmente, se integrará con 
herramientas externas para facilitar su uso con los flujos de trabajo ya 
establecidos.\\

\subsection{Elección del nombre}
\label{subsec:solution:naming}

La herramienta recibirá el nombre de \fronttier. El mismo deviene de un juego 
de palabras entre \emph{front}, nombre que se le suele dar a la 
\viewtier, y \emph{tier}, denominación que recibe cada una de las capas de la 
aplicación.

\subsection{Características deseadas}
\label{subsec:solution:whishlist}

El objetivo del presente trabajo es la generación de una herramienta que se 
posicione como el estándar de los lenguajes de la \jvm en el manejo de 
\dependencies de la \viewtier. Las siguientes características son deseables en 
la herramienta a desarrollar.

\begin{itemize}
	\setlength{\itemsep}{1pt}
	\setlength{\parskip}{0pt}
	\setlength{\parsep}{0pt}
	\item \textbf{Hecho enteramente sobre la \jvm}\\
		Deberá correr enteramente sobre la \jvm, evitando depender de la 
		instalación de software adicional en el equipo y haciéndolo 
		independiente del sistema operativo. Por ésta razón es necesario que se 
		encuentre desarrollada  íntegramente en uno o varios lenguajes que 
		compilen a \bytecode \java.  
	\item \textbf{Posibilidad de ser usado desde cualquier lenguaje de 
	la \jvm}\\
		Deberá poder ser utilizada desde cualquier otro lenguaje de la \jvm de 
		forma natural. En caso de características incompatibles de los 
		lenguajes, debería proveer una capa de abstracción de forma tal que el 
		uso sea transparente y completo. Incluso es deseable que pueda ser 
		ejecutada en forma de programa independiente desde la \cli. Ésto último 
		posibilitaría que pueda ser utilizado prácticamente desde cualquier 
		lenguaje, aunque implica contar con acceso al sistema de archivos del 
		sistema operativo.
	\item \textbf{Archivos de configuración personalizados, con varios 
	formatos por defecto}\\
		La herramienta deberá ser capaz de descargar y acomodar los paquetes 
		que el usuario determine como \dependencies en un \conffile. En lugar 
		de forzar un formato específico para el \conffile, el \depmgr 
		desarrollado deberá brindar al usuario la posibilidad de usar un 
		formato que se adecúe a sus necesidades. Por ejemplo, los usuarios de 
		\apache \maven podrían optar por un archivo de configuración en \xml 
		similar al que acostumbran y conocen, mientras que los usuarios de \sbt 
		podrían elegir un archivo de configuración cuya sintaxis asemeje los 
		usados en esa herramienta. Debería ser lo bastante flexible para 
		permitir la generación de formatos de \conffile por los mismos usuarios 
		de la herramienta de forma sencilla.
	\item \textbf{Múltiples repositorios}\\
		Finalmente, sería ideal contar con un repositorio central, donde 
		desarrolladores puedan subir sus paquetes, o al menos registrar los 
		mismos, permitiendo así consultar la existencia de \dependencies. 
		Además, los usuarios podrían querer generar sus propios repositorios y 
		hacer que la herramienta consulte en ellos, evitando la centralización 
		del repositorio. La dualidad entre contar con un repositorio 
		centralizado, y múltiples repositorios descentralizados, permite a 
		organizaciones de distinto tamaño contar con posibilidades acorde a sus 
		requerimientos, y se ha mostrado exitosa en proyectos como \maven.
	\item \textbf{Seguridad mediante autenticación en los repositorios}\\
		Algunos de éstos repositorios podrían ser privados, y por tanto, se 
		requeriría acceso mediante un usuario y contraseña que la herramienta 
		debe enviar de forma segura.
	\item \textbf{Integración con estándares existentes de \depmgrs}\\
		Al existir soluciones como \emph{Bower} o \emph{Component} la 
		herramienta debería complementarlas, permitiendo, por ejemplo, utilizar 
		sus archivos de configuración y sus repositorios. También es deseable 
		que se integre con los \depmgrs que corren sobre la \jvm, de forma tal 
		que el usuario se pueda despreocupar de tener que correr múltiples 
		programas (el \depmgr para la \logictier y \fronttier para las 
		\dependencies de la \viewtier).
	\item \textbf{Integración con tecnologías web existentes en la \jvm}\\
		Finalmente, al existir numerosos \frameworks para distintos lenguajes 
		de la \jvm que apuntan al desarrollo de aplicaciones web, sería 
		deseable la integración con los mismos. Por ejemplo, múltiples 
		\frameworks utilizan sistemas de templates para generar el código \html 
		que finalmente será utilizado. Sería ideal poder brindar algún 
		componente que integre los elementos descargados a los templates del 
		usuario de forma automática.
\end{itemize}


Se puede agregar además que la mayoría de los \depmgrs actuales se basan en 
archivos, es decir, un archivo por \dependency. En las \dependencies de la 
\viewtier esto no es necesariamente cierto, y a veces una \dependency consiste 
en múltiples archivos. La herramienta tiene que proveer una forma de 
contemplar los casos señalados para permitir bajar todos los archivos que sean 
necesarios.\\

\subsection{Scala como lenguaje de implementación}
\label{subsec:solution:technology}

Se utilizará \scala para desarrollar la aplicación. La elección se basa en que 
la combinación de \emph{Programación Orientada a Objetos} y \emph{Programación 
Funcional} que presenta el lenguaje permiten desarrollar complejas soluciones 
en cortos períodos de tiempo. Además, el código desarrollado es de fácil 
mantenimiento y gran expresibilidad, dado a la capacidad de \scala de crear 
sencillos DSLs.\\
Adicionalmente, el lenguaje elegido presenta una gran cantidad de bibliotecas 
estándar que permiten realizar rapidamente tareas complejas. Por ejemplo, su 
biblioteca de análisis sintáctico (\emph{parsing}) resultará muy útil cuando 
se requiera leer archivos de configuración.\\
\scala será utilizado en combinación con \sbt para manejar las \dependencies del 
proyecto.\\
Al momento de desarrollar herramientas de integración con otros lenguajes o 
tecnologías existentes es posible que sea necesario el uso de otros lenguajes, 
como \java, \clojure o \groovy.


\subsection{Alcance del presente trabajo}
\label{subsec:solution:todo}

El alcance del presente trabajo no abarca la totalidad de las características 
deseadas expresadas en la sección \ref{subsec:solution:whishlist}. El mismo 
solamente se enfocará en las funcionalidades más importantes y  dejará 
asentadas las bases para el desarrollo de las partes no implementadas en 
trabajos futuros. Así, el enfoque será puesto en la estructura del sistema, 
orientada siempre a la extensibilidad del mismo.\\
Se presenta como objetivo el desarrollo de las siguientes 
funcionalidades:
\begin{itemize}
	\setlength{\itemsep}{1pt}
	\setlength{\parskip}{0pt}
	\setlength{\parsep}{0pt}
	\item Desarrollo del núcleo del sistema
	\item Descargar \dependencies desde la web vía \http\footnote{
			HTTP es el protocolo usado en cada transacción de Internet. HTTP 
			define la sintaxis y la semántica que utilizan los elementos de 
			software de la arquitectura web (clientes, servidores, proxies) 
			para comunicarse. Es un protocolo orientado a transacciones y sigue 
			el esquema petición-respuesta entre un cliente y un servidor.
		}
	\item Descargar desde \emph{GitHub}\footnote{
			GitHub es un popular sitio web que provee repositorios \git 
			gratuitos para proyectos de código abierto.
		} mediante \git\footnote{
			GIT es un software de control de versiones distribuido, pensando en 
			la eficiencia y la confiabilidad del mantenimiento de versiones de 
			aplicaciones cuando	estas tienen un gran número de archivos de 
			código fuente.
		}
	\item Descargar desde un repositorio \svn\footnote{
			Subversion (SVN) es una herramienta de control de versiones open 
			source basada en un repositorio cuyo funcionamiento se asemeja 
			enormemente al de un sistema de ficheros.
		}
	\item Lectura de \conffiles en \xml\footnote{
			XML es un lenguaje de marcas utilizado para almacenar datos en forma
			legible. Se propone como un estándar para el intercambio de 
			información estructurada entre diferentes plataformas.
		} estilo \apache \maven
	\item Lectura de \conffiles en \xml estilo \apache \ivy
	\item Agregado de múltiples repositorios
	\item Usar la herramienta desde la \cli
	\item Usar la herramienta desde \scala
	\item Integración de la herramienta con \sbt
	\item Integración de la herramienta con Play! Framework
\end{itemize}

La herramienta será liberada como \freesoft con licencia \apache versión 2. 
Esto permitirá que la comunidad de desarrolladores la expanda para  
funcionar con todos los lenguajes de la \jvm, en muchos más \frameworks, y que  
se integre con una mayor cantidad de procesos.