\setcounter{figure}{0}
\renewcommand{\figurename}{UML}

\section{Desarrollo de \fronttier}

El sistema está compuesto por dos módulos independientes, \scaliapp y 
\scalavcs, y por el código principal de \fronttier. \scaliapp se centra en 
permitir desarrollar aplicaciones que corren desde la \cli de forma rápida y 
sencilla, mientras que \scalavcs brinda una capa de abstracción a los 
repositorios de control de versiones. Ambas bibliotecas son utilizadas desde 
\fronttier, pero pueden también ser utilizadas en otros proyectos.\\
El código desarrollado es lo más simple posible para una mayor escalabilidad y 
mejor mantenimiento a futuro. Todas las características avanzadas de \scala 
fueron usadas en la implementación.\\
El código y la documentación detallada para cada uno de los objetos, clases y 
métodos se encuentra disponible para su análisis en el sitio web de \fronttier 
(\emph{http://fronttier.alanrodas.com}).

\subsection{Patrones de diseño y desarrollo en \scala}

Al desarrollar sistemas se utilizan generalmente una serie de soluciones 
reutilizables para problemas comunes, las cuales se denominan \emph{patrones de 
diseño}. En algunos casos, los \emph{patrones de diseño} son una señal de 
una característica faltante en el lenguaje de programación usado. El diseño del 
lenguaje \scala soluciona por sí mismo muchos de los patrones clásicos, tanto 
de la \emph{Programación Orientada a Objetos} \citesq{Gamma:1995:BOOK}, como de 
la \emph{Programación Funcional} \citesq{Yorgey:2009:ONLINE}.\\
Así, el patrón \emph{singleton} se utilizó en varias oportunidades mediante la 
creación de objetos con \code{object} y \emph{factory methods} fueron 
utilizados mediante el uso de \code{apply} como constructores. Las conversiones 
implícitas en el lenguaje permitieron implementar el patrón \emph{adapter}, 
mientras que utilizar el patrón \emph{value object} fue trivial gracias al uso 
de \code{case classes}. Los patrones \emph{command}, \emph{strategy} y 
\emph{chain of responsibility} se lograron mediante la posibilidad de usar 
funciones como elementos de primer nivel que pueden pasarse por parámetro, o la 
posibilidad de tener funciones parciales. El patrón \emph{null object} se logró 
mediante el uso de \code{Option}, aunque la funcionalidad no es exactamente 
igual al patrón tradicional. Por último, el patrón \emph{dependency injection} 
se logra mediante el uso de \code{traits} y la capacidad de asociarlos a un 
objeto particular en lugar de a una clase, algo conocido como \emph{cake 
pattern} \citesq{Bonér:2008:ONLINE}.\\
En cuanto a los patrones de diseño de lenguajes funcionales, \scala provee 
\emph{functores} y \emph{monadas} en clases como \code{Option} y \code{Either} o \code{Error}. Estas clases fueron ampliamente utilizadas en el código de \fronttier, de \scalavcs y \scaliapp. Además, \scala provee \emph{pattern matching}, \emph{inferencia de tipos} y  otras características que suelen ser comunes de lenguajes funcionales.\\
Finalmente, \scala da la posibilidad de utilizar métodos que toman un solo 
argumento de forma infija o llamar métodos sin utilizar paréntesis. Ésto, junto 
con las caracteristicas de \code{package objects} y el patrón \emph{builder}, 
permite crear lenguajes de dominio específico (DSLs). Esta técnica es 
ampliamente utilizada para proveer una interfaz más clara y concisa a los 
desarrolladores o incluso a clientes como explica 
\citeauthor{Venners:2009:ONLINE} en \citetitle{Venners:2009:ONLINE}, y fue 
usada en \fronttier y las bibliotecas para dar una mayor expresibilidad al 
código a usar. 

\subsection{Modelo del parser de \cli}

El módulo \scaliapp es una biblioteca para crear aplicaciones de \cli y que 
funciona de forma completamente independiente a \fronttier.\\
La biblioteca permite extender la clase \code{CLIApp} para crear aplicaciones 
capaces de ser ejecutadas desde la \cli recibiendo argumentos, flags y 
comandos. Para ello, \scaliapp provee un DSL que ayuda a especificar los 
comandos que serán válidos para la aplicación, que argumentos podrá recibir, y 
qué acción se ejecutará una vez recibido dicho comando.\\
El código principal se encuentra en el paquete \code{com.alanrodas.scaliapp}, 
el cual, al importarlo, provee todas las clases y funciones necesarias para 
utilizar el DSL. Además provee la clase \code{CLIApp}. La misma debe ser 
extendida por el usuario en el objeto que declarará en su cuerpo los comandos 
válidos de la aplicación. \code{CLIApp} se encargará de ejecutar el código 
correspondiente al \code{callback} del comando ejecutado al invocarse la 
aplicación.\\
El UML en la \figref{uml:scaliapp:general} provee una vista simplificada del 
modelo de definiciones de \scaliapp. \code{CLIApp} no es más que una clase que 
provee las mismas funcionalidades que \code{App} en \scala, pero evaluando 
todos los comandos declarados y delegando su procesamiento en 
\code{CommandManager}. Para cada \code{CommandDefinition} existe un 
\code{callback}, una función que será ejecutada cuando se invoque el comando 
correspondiente, y que recibirá los datos que se hayan pasado al comando. Estas 
funciones no tienen valor de retorno y se espera que ejecuten alguna acción 
concreta.\\

\begin{figure}[htb]
\tiny
\centering{
\begin{tikzpicture}
	\begin{object}[text width=5cm]{CLIApp}{0,18}
		\operation{main(args : Array[String]) : void}
	\end{object}
	
	\begin{object}[text width=7cm]{CommandManager}{9,18}
		\operation{execute(args : Array[String])}
	\end{object}
	
	\aggregation{CLIApp}{manager}{1..1}{CommandManager}
	
	\begin{abstractclass}[text width=5cm]{CommandDefinition}{9,16}
		\attribute{name : String}
	\end{abstractclass}
	
	\aggregation{CommandManager}{commands}{1..*}{CommandDefinition}
	
	\begin{class}[text width=5cm]{FiniteValuesCommand}{4,12}
		\inherit{CommandDefinition}
		\attribute{callback : FiniteValuesCommand => Unit}
	\end{class}
	
	\begin{class}[text width=6cm]{MultipleValuesCommand}{10,12}
		\inherit{CommandDefinition}
		\attribute{callback : MultipleValuesCommand => Unit}
	\end{class}
	
	\begin{abstractclass}[text width=4cm]{ParameterDefinition}{2,16}
		\attribute{value : String}
		\attribute{name : String}
		\attribute{alias : Option[String]}
	\end{abstractclass}
		
	\begin{class}[text width=3cm]{ArgumentDefinition}{0,14}
		\inherit{ParameterDefinition}
	\end{class}
	
	\begin{class}[text width=3cm]{FlagDefinition}{4,14}
		\inherit{ParameterDefinition}
	\end{class}
	
	\unidirectionalAssociation{CommandDefinition}{}{}{ParameterDefinition}
	
	\begin{class}[text width=3cm]{ValueDefinition}{4,10}
		\attribute{value : String}
		\attribute{name : String}
		\attribute{alias : Option[String]}
	\end{class}
	
	\aggregation{FiniteValuesCommand}{values}{1..*}{ValueDefinition}
	
\end{tikzpicture}
}
\normalsize
\caption{Diagrama general de \scaliapp.}
\label{uml:scaliapp:general}
\end{figure}

Al ser una biblioteca completamente independiente del proyecto \fronttier, 
\scaliapp tiene sus propios objetivos, trabajos a futuro, etc. En particular, 
la idea general es que la herramienta sea lo suficientemente flexible como para 
soportar cualquier aplicación que corra desde la \cli, incluyendo aplicaciones 
interactivas. La documentación puntual, así como la información de tareas a 
realizar y el código actualizado puede encontrarse en la web de \scaliapp 
(http://scaliapp.alanrodas.com).\\
Según nuestro punto de vista el trabajo realizado para \scaliapp no se 
considera terminado pues solamente se limitó al desarrollo de la parte que 
fuera imprescindible para \fronttier. Como trabajo a futuro en este ámbito, se 
plantea el soporte de aplicaciones interactivas, mayor formato de comandos, 
soporte multilenguaje y uso y generación de archivos de ayuda.\\
Como aplicación que corre desde la \cli, \fronttier hace uso de \scaliapp para 
determinar todos los comandos que serán válidos para la aplicación.

\subsection{Modelo de \conffiles}

Los \conffiles de \fronttier se leen en base a una serie de objetos registrados 
en el sistema. Cada uno de dichos objetos es instancia de una clase que 
extiende el trait \code{ConfigParser} y reimplementa el método \code{parse}, 
así como el valor \code{id}, entre otras propiedades.\\
Todas las clases que implementan el trait \code{ConfigParser} son identificadas 
automáticamente utilizando \emph{reflection} al comenzar la ejecución del 
programa. Para cada clase, una nueva instancia de la misma es creada y 
registrada en el sistema como encargada de analizar el \conffile que le 
compete.\\
\fronttier permite cargar paquetes JAR creados por terceros. Esto posibilita 
cargar nuevos \emph{parsers} que son luego registrados automáticamente. Así, 
los desarrolladores que quieran extender la herramienta de esta forma solamente 
requerirán crear una clase que implemente \code{ConfigParser}.\\
A lo largo de \fronttier, las instancias creadas al analizar los \emph{parsers} 
son pasadas en un parámetro implícito. Además, estas instancias son creadas en 
forma \emph{lazy}, de forma tal que se creen solamente objetos que van a ser 
efectivamente utilizados, mejorando la performance de la aplicación.

\subsection{Modelo de sistema de descargas}

El modelo del sistema de descargas cuenta con dos partes: la parte de descargas 
mediante \http y que se encuentra dentro del modulo de \fronttier; y la parte 
de descargas desde sistemas de control de versiones, el cual hace uso del 
modulo \scalavcs.\\
Para descargar archivos mediante \http, \fronttier utiliza la biblioteca 
\code{RaptureIO}, la cual provee una base de abstracción sobre el proceso de 
entrada y salida en \scala. De esta forma, se puede descargar un archivo de 
\internet y guardarlo en el disco duro como si el mismo estuviera en la máquina 
local, sin preocuparse por particularidades del sistema operativo o el sistema 
de archivos.\\
Por otro lado, \fronttier utiliza \scalavcs para descargar archivos desde 
sistemas de control de versiones (VCSs) como \git, \svn, \cvs,\bazaar y 
\mercurial.\\
\scalavcs provee una capa de abstracción sobre los distintos sistemas de 
control de versiones, partiendo de la base que existen una serie de acciones 
que el usuario puede desear realizar sobre un VCSs. Tales operaciones incluyen 
añadir un archivo al sistema de control, remover un archivo, listar los 
archivos agregados, salvar la versión actual, volver a una versión anterior, 
ver las versiones salvadas o comparar dos versiones salvadas, entre otras.\\
Así, \scalavcs provee clases que representan a cada uno de los posibles 
sistemas de control de versiones, las cuales poseen métodos polimórficos para 
dichas operaciones. Todo esto se acompaña de un DSL que permite ejecutar código 
como el mostrado en el ejemplo \ref{development:scalavcs:sample}. El mismo 
asemeja en gran medida los comandos que este tipo de herramientas proveen desde 
la \cli.\\

\begin{listing}[ht]
	\begin{minted}[frame=lines]{scala}
git = GIT clone "http://myrepo/mycoolproject"
git rm "someFile"
git add "*"
git commit "Some message"
	\end{minted}
	\caption{Ejemplo del formato \emph{fronttier}}
	\label{development:scalavcs:sample}
\end{listing}

A su vez, \scalavcs distingue dos tipos de sistemas de control de versiones:
aquellos que funcionan con un solo servidor (\cvs, \svn) y los distribuidos 
(\git, \mercurial, \bazaar). Los sistemas distribuidos tienen acciones 
adicionales, como sincronizar de forma distribuida, o bajar una versión 
específica.\\
El diagrama UML \ref{uml:scalavcs:general} muestra la estructura básica de los 
repositorios de \scalavcs. Existe, a su vez, un objeto para cada tipo de 
sistema de control de versiones que actúa a modo de \emph{factory}.\\

\begin{figure}[htb]
\tiny
\centering{
	\begin{tikzpicture}
	\begin{abstractclass}[text width=5cm]{VCSRepo}{6,8}
		\operation{add(resource : String) : Boolean}
		\operation{rm(resource : String) : Boolean}
		\operation{mv(resource : String) : Boolean}
		\operation{lock(resource : String) : Boolean}
		\operation{status() : VCSStatus}
		\operation{diff() : VCSDiff}
		\operation{branch(name : String) : Boolean}
		\operation{checkout(version : String) : Boolean}
		\operation{export(path : String) : Boolean}
		\operation{commit(msg : String) : Boolean}
	\end{abstractclass}
	
	\begin{abstractclass}[text width=4cm]{SimpleVCSRepo}{2,3}
		\inherit{VCSRepo}
	\end{abstractclass}

		\begin{class}[text width=2cm]{SVNRepo}{0,0}
		\inherit{SimpleVCSRepo}
		\end{class}
		
		\begin{class}[text width=2cm]{CVSRepo}{4,0}
		\inherit{SimpleVCSRepo}
		\end{class}
		
	\begin{abstractclass}[text width=4cm]{DistributedVCSRepo}{10,3}
		\inherit{VCSRepo}
		\operation{pull(repo : String) : Boolean}
		\operation{push(repo : String) : Boolean}
		\operation{fetch(repo : String) : Boolean}
	\end{abstractclass}

		\begin{class}[text width=2cm]{GitRepo}{7,0}
		\inherit{DistributedVCSRepo}
		\end{class}
		
		\begin{class}[text width=2cm]{HgRepo}{10,0}
		\inherit{DistributedVCSRepo}
		\end{class}
		
		\begin{class}[text width=2cm]{BzRepo}{13,0}
		\inherit{DistributedVCSRepo}
		\end{class}
			
	\end{tikzpicture}
}
\normalsize
\caption{Diagrama general de los repositorios de \scalavcs.}
\label{uml:scalavcs:general}
\end{figure}

\scalavcs utiliza parámetros implícitos en todos los objetos que actúan de 
\emph{factory}. Estos parámetros pueden ser importados desde el paquete
\code{com. alanrodas.scalavcs.connectors} y sus subpaquetes, pasándose 
automáticamente a los objetos constructores. Tienen por objetivo proveer el 
\emph{conector} que se utilizará como cliente para un determinado sistema de 
VCS. Así, es posible usar un conector que utilice la aplicación \svn mediante 
llamadas a la misma a través del sistema operativo, o utilizar una 
implementación desarrollada completamente en \java, como \emph{SVNKit}.\\
\scalavcs también es un proyecto independiente de \fronttier. El desarrollo 
realizado en esta herramienta es mínimo, por lo cual, no se ha lanzado ninguna 
versión al momento de publicar el presente documento, quedando como trabajo a 
futuro terminar la implementación para todos los VCSs. Puede seguirse el 
desarrollo y planes a futuro del proyecto en el sitio web de \scalavcs 
(http://scalavcs.alanrodas.com).

\subsection{Modelo general}

Las partes anteriores se combinan finalmente en \fronttier dando lugar a la 
aplicación prácticamente completa. Las clases simples que representan los 
conceptos básicos de \emph{repositorios}, \dependencies y otros, complementan 
algunos patrones que cierran la aplicación. A esto, se suma al desarrollo de la 
aplicación el agregado de \emph{logging}, el \emph{testing}, y los \emph{Scala 
Docs} necesarios para una correcta documentación.\\
Cabe mencionar que se utilizó \git como sistema de control de versiones en el 
proyecto, \sbt como manejador de dependencias del mismo e \emph{IntelliJ IDEA} 
como IDE de desarrollo. El sitio web de la herramienta fue desarrollado 
utilizando \htmlv, con \js y \css.\\
Por último, este documento, así como otros archivos de documentación en el 
sitio, fueron creados con \emph{XeLaTeX}, \emph{Biber} y la herramienta 
\emph{arara}\footnote{
	\emph{arara} es una herramienta programada en Perl que permite simplificar 
	el proceso de compilación de documentos de \emph{LaTeX}, \emph{XeLaTeX} y 
	otros derivados de \emph{TeX} mediante comentarios en el documento.
} para su compilación. La versión de \emph{TeX} utilizada corresponde a 
\emph{TeXLive 2014}, con el paquete \emph{minted} en su versión de desarrollo, 
y la herramienta de \emph{Python} \emph{pygmentize} en su versión 1.6. 
\emph{TeX Studio} fue usado como IDE para los documentos.
