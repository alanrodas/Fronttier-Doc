\section{Trabajo a futuro}
\label{sec:futurework}

Como se refleja en la sección \ref{subsec:solution:todo}, este trabajo tiene un 
alcance limitado con respecto a los deseos expresados en la sección 
\ref{subsec:solution:whishlist}. En particular quedan expresadas pero 
pendientes de desarrollo las siguientes tareas:

\begin{itemize}
	\setlength{\itemsep}{1pt}
	\setlength{\parskip}{0pt}
	\setlength{\parsep}{0pt}
	\item Soportar más formas de descarga a través de otros métodos sistemas de
		control de versiones, como \emph{Mercurial}, \emph{Bazaar}, \emph{CVS}, etc.
	\item Mayor cantidad de formatos de \conffiles
	\item Seguridad en los repositorios mediante autenticación
	\item Funcionar desde otros lenguajes de la \jvm como \clojure y \groovy
	\item Integración con más cantidad de procesos, como \emph{Grape}, \emph{Gradle}, etc.
	\item Integración con mayor cantidad de \frameworks web
	\item Integración con interfaces de desarrollo integradas (IDEs)
\end{itemize}

También queda pendiente la generación de un sitio capaz de funcionar como
repositorio de \dependencies centralizado para que se constituya como la
referencia en cuanto a paquetes para usar en la \viewtier. El sitio debería
permitir a los desarrolladores subir y manejar ellos mismos los paquetes que
desean distribuir en el repositorio. Este es realmente uno de los trabajos 
pendientes más importantes para lograr una buena aceptación de la herramienta y 
posicionarla como un estándar en la industria.\\
El hecho de que exista una enorme cantidad de lenguajes para la \jvm, y una 
enorme cantidad de \frameworks especializados en el desarrollo de aplicaciones 
web que corren en los mismos, hace que sea extremadamente sencillo encontrar 
constantemente nuevas posibilidades a realizar. La adaptación a más lenguajes 
de forma \quoted{nativa}, y la integración con más procesos y \frameworks es 
una de las formas de expansión de este trabajo más deseables.\\
Finalmente podemos comentar que si bien la plataforma elegida fue la \jvm,
existen numerosos lenguajes que no corren sobre esta y que se enfocan también
en el desarrollo de sistemas web, como la plataforma \emph{.NET} o \emph{PHP}.
Se podrían entonces crear adaptaciones (\emph{ports}) a estos otros lenguajes y
plataformas como una opción de trabajo a futuro.\\
Por último, es de suma importancia remarcar que la herramienta es liberada como
\freesoft haciendo que el trabajo pendiente expresado en esta sección pueda ser
realizado por la comunidad. Incluso es posible que existan desarrollos que 
expandan el sistema en formas que no han sido planteadas en esta sección.\\