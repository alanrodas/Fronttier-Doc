\section{Conclusiones y trabajos a futuro}
\label{sec:futurework}

Dentro del contexto de los \depmgrs para la \jvm, \fronttier innova por 
ser la primera en encargarse de los recursos web de un sistema, así como 
también lo hace en contexto de los \depmgrs para la \viewtier por ser el 
primero pensado para la plataforma \java.\\
\fronttier permite al usuario descargar \dependencies web de forma sencilla, 
desde la \cli, y desde aplicaciones que corren sobre la \jvm. Se 
ha integrado a procesos existentes de desarrollo de software para facilitar su 
adopción.\\
Se ha logrado construir una herramienta que es altamente extensible, y que pone 
al desarrollador en una posición de pocas restricciones con respecto a las 
extensiones posibles a la misma.\\
Si bien la herramienta puede no lograr convertirse en un estándar, no es 
difícil pensar que futuras modificaciones como, mayor cantidad de interacción 
con otras herramientas y la creación de repositorios libres y comunes puedan 
llevar a \fronttier a convertirse en un referente en cuanto a manejo de 
dependencias de la \viewtier se refiere.\\
Durante el proceso de programación de la herramienta, se ha podido observar 
como el desarrollo de módulos pensados como una API son fundamentales para una 
fácil y rápida extensión del sistema.\\

\jump[2]

Como se observa en la sección \ref{subsec:solution:todo}, el presente trabajo 
tiene un alcance limitado con respecto a los deseos expresados en la sección 
\ref{subsec:solution:whishlist}. En particular, sería interesante desarrollar 
las siguientes extensiones:

\begin{itemize}
	\setlength{\itemsep}{1pt}
	\setlength{\parskip}{0pt}
	\setlength{\parsep}{0pt}
	\item El soporte de otras formas de descarga, por ejemplo, con otros 
		sistemas de control de versiones como \emph{Mercurial}, \emph{Bazaar}, 
		\emph{CVS}, etc.
	\item El uso de una mayor cantidad de formatos de \conffiles
	\item La seguridad en los repositorios mediante autenticación
	\item El funcionamiento desde otros lenguajes de la \jvm como \clojure y 
	\groovy
	\item La integración con más cantidad de procesos, como \emph{Grape}, 
		\emph{Gradle}, etc.
	\item La integración con mayor cantidad de \frameworks web
	\item La integración con interfaces de desarrollo integradas (IDEs)
\end{itemize}

También resulta interesante como extensión la generación de un sitio que 
funcione como repositorio de \dependencies centralizado para que se constituya 
como la referencia en cuanto a paquetes para usar en la \viewtier. El sitio 
debería permitir a los desarrolladores subir y manejar ellos mismos los 
paquetes que desean distribuir en el repositorio. Este es realmente uno de los 
trabajos pendientes más importantes para lograr una buena aceptación de la 
herramienta y posicionarla como un estándar en la industria.\\
El hecho de que exista una enorme cantidad de lenguajes para la \jvm, y una 
enorme cantidad de \frameworks especializados en el desarrollo de aplicaciones 
web que corren en los mismos, hace que sea extremadamente sencillo encontrar 
constantemente nuevas posibilidades a implementar. La adaptación a más 
lenguajes de forma \quoted{nativa}, y la integración con más procesos y 
\frameworks es una de las formas de expansión de este trabajo más deseables.\\
Finalmente, si bien la plataforma elegida fue la \jvm,
existen numerosos lenguajes que no corren sobre ésta y que se enfocan también
en el desarrollo de sistemas web, como la plataforma \emph{.NET} o \emph{PHP}.
Se podrían entonces crear adaptaciones (\emph{ports}) a dichos lenguajes y
plataformas como una opción de trabajo a futuro.\\
Por último, es de suma importancia remarcar que la herramienta es liberada como
\freesoft haciendo que el trabajo pendiente expresado en esta sección pueda ser
realizado por la comunidad. Incluso es posible que existan desarrollos que 
expandan el sistema en formas que no han sido planteadas en esta sección.\\