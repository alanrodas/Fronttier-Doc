\section{Guía de Uso}
\label{sec:guide}

El presente capítulo muestra el funcionamiento de \fronttier, el
sistema de manejo de dependencias para la \viewtier que se ha
desarrollado.\\
Se explica entonces su uso desde la interfaz de \cli, su integración
y uso desde sistemas en distintos lenguajes
de programación y otros aspectos de la aplicación.\\

\subsection{Uso desde la \cli}
\label{subsec:guide:cli}

Como aplicación que corre desde la \cli, \fronttier es
extremadamente sencilla de utilizar, si bien posee un amplio número
de comandos y configuraciones.\\
Para correr la aplicación, es necesario ejecutar el archivo \jar
de la herramienta, a través de la \jvm.
El siguiente código muestra como ejecutar \fronttier desde la interfaz
de \cli:
\cmd{java -jar fronttier.jar}
Adicionalmente, se puede pasar una serie de argumentos a la aplicación
para configurarla a gusto y necesidad del usuario, tal como se muestra
a continuación:
\cmd{java -jar fronttier.jar <argumentos>}
Con intención de simplificar el trabajo del usuario al momento de ejecutar la
aplicación a través de la \cli, \fronttier incluye
un simple programa que puede ser colocado en cualquier lugar accesible
para el usuario\footnote{
	Un lugar accesible es, por ejemplo, cualquier carpeta que se
	encuentre dentro de la variable de entorno \quoted{PATH} en el sistema
	operativo del usuario.
}, permitiendo ejecutar la aplicación de forma sencilla, tal como
se muestra en el siguiente ejemplo:
\cmd{fronttier <argumentos>}
En los próximas apartados se verán los argumentos posibles en más
detalle.

\subsubsection{Uso con los valores por defecto}
\label{subsubsec:guide:defaults}

Al ser ejecutada sin ningún tipo de argumentos, \fronttier utilizará los
valores por defecto de la aplicación. Estos incluyen: utilizar la
\cacheg, buscar dependencias en un archivo de
configuración con el formato estándar de \fronttier con el nombre
\emph{fronttier.ftt} e instalar las dependencias descargadas en el directorio
local.\\
Es decir, es posible llamar al programa sin ningún argumento y obtener
funcionalidad del mismo mediante \quoted{convenciones} que \fronttier
utiliza.\\
En las próximas secciones se verá con mayor profundidad cómo funcionan
cada uno de los argumentos y a qué refieren los valores por defecto con
mayor precisión.

\subsubsection{Selección del archivo de configuración y su formato}
\label{subsubsec:guide:confformat}

Por defecto, \fronttier intentá localizar en el directorio actual un
archivo con el nombre de \emph{fronttier.ftt}. Este archivo es donde
se declaran las dependencias y repositorios del \project.\\
A su vez, el usuario puede especificar un archivo distinto en donde
buscar dependencias. Esto se logra mediante la opción \ddashcmd{filename}
o \dashcmd{f} pasada como argumento a la aplicación e incluyendo luego el nombre del
archivo que se desea utilizar para declaración de dependencias
y repositorios. A continuación se puede observar un ejemplo de como
llamar a la aplicación para que lea un archivo llamado \emph{dependencies.ftt}:
\cmd{fronttier --filename dependencies.ftt}
Estos archivos pueden tener internamente distintos formatos, es decir, distintas
formas de representar la información. Cada formato posee un nombre único por el
cual el usuario puede identificarlo y, así, indicarle a la herramienta qué debería
esperar de
los contenidos del archivo de configuración. Esto se logra mediante el
argumento \ddashcmd{config} o \dashcmd{c}. El
siguiente código muestra como elegir un formato con nombre \quoted{xml}:
\cmd{fronttier --config xml}
Los formatos registrados por defecto en \fronttier son:
\begin{itemize}
	\setlength{\itemsep}{1pt}
	\setlength{\parskip}{0pt}
	\setlength{\parsep}{0pt}
	\item fronttier
	\item xml
	\item attrxml
	\item scale
\end{itemize}
Sin embargo, el usuario puede registrar nuevos formatos mediante extensiones
a la herramienta. En caso de no especificar el formato del archivo, el
formato \emph{fronttier} es el utilizado.\\
El formato elegido del archivo repercute sobre el nombre por defecto del
mismo. Por ejemplo, el nombre de archivo por defecto para el
formato \emph{xml} es \emph{fronttier.xml}, al igual que para el formato
\emph{attrxml}. Por su parte el formato \emph{fronttier} buscará
un archivo por el nombre de \emph{fornttier.ftt}.\\
Los argumentos \ddashcmd{config} y \ddashcmd{filename} se pueden usar al mismo
tiempo dando lugar a cualquier combinación de nombre de archivo y formato.
Por ejemplo, el siguiente código muestra como utilizar \fronttier para
que utilice un archivo \emph{dependencies.xml} que posee formato
\emph{xml}.
\cmd{fronttier --config xml --filename dependencies.xml}
La estructura interna de cada uno de los formatos, así como la forma
de declarar dependencias y repositorios es especificada en la
sección \ref{subsec:formats}.

\subsubsection{Elección de carpeta de destino}
\label{subsubsec:guide:folder}

Por defecto \fronttier trabaja sobre la carpeta actual, y es allí en
donde se \emph{instalan} las dependencias descargadas.\\
No obstante, la carpeta sobre la que trabajará la herramienta puede
ser especificada, de forma tal que las dependenciias se instalen en alguna carpeta
especifica o una subcarpeta de la carpeta local. Para ello se
utiliza el comando \ddashcmd{destination} o \dashcmd{d} seguido de la ubicación
en donde se desean descargar las dependencias.
\cmd{fronttier --destination <folder>}
El valor por defecto es \emph{.}, es decir, la carpeta
actual. Por tanto, los siguientes dos comandos son equivalentes
\cmd{fronttier}
\cmd{fronttier --destination .}
El elegir la carpeta de destino resulta útil en proyectos en donde
el código compilado termina en una carpeta especifica que no forma
parte del proyecto y que se elimina cada vez que se recompila el mismo.
Lo más común es, entonces, elegir un subdirectorio de la carpeta de trabajo
en donde se deben colocar las dependencias previo a ejecutar el sistema.
El siguiente ejemplo muestra como indicarle a \fronttier que utilice
como carpeta de descarga un subdirectorio con el nombre de
\quoted{WEB-INF/html}:
\cmd{fronttier --destination ./WEB-INF/html}
Si bien no es algo común, se pueden especificar también direcciones
absolutas en lugar de relativas. En caso de que las direcciones
contengan espacios se pueden utilizar comillas simples o dobles para
escapar la dirección como se muestra en el siguiente ejemplo.
\cmd{fronttier --destination "./Web Dependencies"}
Se debe tener en cuenta que \fronttier no realiza ninguna
acción para limpiar la carpeta de destino, con lo que el
usuario deberá garantizar su limpieza previo a futuras
ejecuciones.

\subsubsection{Cache local y cache global}
\label{subsubsec:guide:localmode}

\fronttier mantiene en la maquina del usuario una cache con
todos los archivos que ha descargado al momento. Esto permite
acelerar el proceso de \emph{instalación} de las dependencias
en un proyecto  particular. Así, \fronttier mantiene dos lugares
que utiliza de cache, la \cachel y la \cacheg. La \cachel tiene
alcance solo sobre el proyecto actual sobre el cual se está trabajando
Su uso se limita a evitar tener que descargar las dependencias en subsequentes
ejecuciones del programa del usuario. Por su parte, la \cacheg
contiene las dependencias descargadas por todos los proyectos, y su uso
es, no solo evitar la descarga de dependencias en futuras ejecuciones del
programa, sino además evitar la descarga en caso de que dos proyectos
independientes requieran eventualmente la misma dependencia. En ambos
casos, además, permiten al desarrollador trabajar sin la necesidad de
una conexión a Internet.\\
La carpeta de \cacheg se encuentra en una carpeta especial creada
por \fronttier en la carpeta personal del usuario, mientras que la \cachel
se encuentra en una carpeta similar, pero localizada en el directorio
del proyecto sobre el cual tiene injerencia. Las estructuras de carpetas
expresadas en \ref{folders:repos:linux}, \ref{folders:repos:windows} y
\ref{folders:repos:osx} muestran la ubicación de la carpeta de cache
en distintos sistemas operativos.

\begin{folders}[h]
	\dirtree{%
		.1 /.
			.2 home.
				.3 username.
					.4 .ftt.
						.5 Repository\DTcomment{Cache Global}.
					.4 proyectos.
						.5 miProyecto.
							.6 .ftt.
								.7 Repository\DTcomment{Cache Local}.
					.4 \ldots.
			.2 \ldots.
	}
	\caption{Caches en un sistema Linux}
	\label{folders:repos:linux}
\end{folders}

\begin{folders}[h]
	\dirtree{%
		.1 C:/.
			.2 Users.
				.3 username.
					.4 .ftt.
						.5 Repository\DTcomment{Cache Global}.
					.4 proyectos.
						.5 miProyecto.
							.6 .ftt.
								.7 Repository\DTcomment{Cache Local}.
					.4 \ldots.
			.2 \ldots.
	}
	\caption{Caches en un sistema Windows}
	\label{folders:repos:windows}
\end{folders}

\begin{folders}[h]
	\dirtree{%
		.1 /.
			.2 Users.
				.3 username.
					.4 .ftt.
						.5 Reporitory\DTcomment{Cache Global}.
					.4 proyectos.
						.5 miProyecto.
							.6 .ftt.
								.7 Repository\DTcomment{Cache Local}.
					.4 \ldots.
			.2 \ldots.
	}
	\caption{Caches en un sistema Mac OS X}
	\label{folders:repos:osx}
\end{folders}

Al momento de instalar una dependencia, \fronttier debe buscar la
misma en algún sitio previo a copiarla (o descargarla y copiarla)
en la carpeta de destino. Al analizar cada una de las dependencias
requeridas, \fronttier evalúa en primer lugar si la misma se encuentra
en la \cachel, en caso de no encontrarla buscará luego en la \cacheg. De
no encontrar la dependencia en ninguna de las caches, recurrirá a buscar
en los \onlinerepo especificados y en los por defecto. Si finalmente no
le es posible encontrarla en dichos repositorios, la herramienta le
notificará al usuario lo ocurrido con un mensaje de error.\\
Por defecto, si una dependencia requerida no se encuentra en la cache
(inicialmente vacía) \fronttier la descargará de los \onlinerepo 
y la copiará a la \cacheg para futuro uso. Este comportamiento esta
dado mediante el flag \ddashcmd{global} o \dashcmd{g}. El comportamiento
puede ser modificado mediante el flag \ddashcmd{local} o \dashcmd{l} para
descargar las dependencias a la \cachel en lugar de la global, o
\ddashcmd{nocache} para no utilizar la cache en absoluto. A su vez, se
puede utilizar el flag \dashcmd{f} o \ddashcmd{force} para forzar la
descarga de la dependencia desde los repositorios en linea incluso si
se encuentran en la cache. Esta acción reemplazará la versión actual en
la cache con la nueva versión.\\
El siguiente código muestra como descargar dependencias sin guardarlas
en la cache, y utilizando siempre la versión disponible en los
\onlinerepo.
\cmd{fronttier --nocache -f}
El siguiente, como descargar las dependencias en la \cachel,
independientemente de si se encuentran en la alguna de las caches.
\cmd{fronttier -lf}
Como se ve en este último comando, los flags en su formato corto pueden
agruparse utilizando un solo signo - y pasando todos los flags
juntos. Para este caso \dashcmd{lf} es equivalente a \dashcmd{l} \dashcmd{f}.


\subsubsection{Modo verbose}
\label{subsubsec:guide:verbose}

Por defecto \fronttier brinda muy poca información al usuario con respecto a
que está haciendo en cada momento, indicando solamente los errores
que se puedan llegar a producir. Es posible, sin embargo, cambiar dicho
comportamiento activando el modo \quoted{verbose}.\\
Precisamente el flag \ddashcmd{verbose} o \dashcmd{v} activa este modo que permite
identificar con mayor precisión que es lo que se encuentra realizando la
herramienta en cada etapa del proceso de resolución de dependencias.\\


\subsubsection{Agregar archivos de expansión}
\label{subsubsec:guide:expand}

Como ya se ha expresado en este documento, la herramienta \fronttier puede ser
extendida por código del usuario. Así, es necesario indicarle a la misma de
alguna forma que archivos se desean cargar\footnote{
	Los archivos a cargar consisten en archivos \jar (código compilado a
	\bytecode \java)	
}. Esto es posible gracias al argumento \ddashcmd{load} o \dashcmd{l} el cual
recibe como argumento el o los nombres de archivo a cargar. Por ejemplo, para
cargar los archivos \quoted{first.jar} y \emph{second.jar} se debería ejecutar
a la aplicación de la siguiente forma
\cmd{fronttier --load first second}
La extensión del archivo no es necesaria pues se asume que siempre será .jar.\\
Los archivos de extensión, en este caso, se encuentran en la carpeta actual
desde donde se ejecuta la aplicación, pero se pueden utilizar también rutas
absolutas o relativas para llamar a los archivos
\cmd{fronttier --load "C:\Program Files\Fronttier Extensions\first"}
Como se puede ver, se puede entrecomillar la ruta en caso de que la misma
contenga espacios.\\
Finalmente, cabe destacar la posibilidad de \emph{instalar} extensiones en
una carpeta global o local de forma tal que puedan ser utilizados en cualquier
momento. Esta carpeta actúa de forma similar a la cache para los archivos
de dependencia. De hecho, su ubicación es precisamente en un directorio hermano
a la carpeta del repositorio de cache. En las estructuras de carpeta
\ref{folders:plugins:linux}, \ref{folders:plugins:windows} y \ref{folders:plugins:osx}
se muestra la ubicación de las carpetas de extensiones de \fronttier.


\begin{folders}[h]
	\dirtree{%
		.1 /.
			.2 home.
				.3 username.
					.4 .ftt.
						.5 Plugins\DTcomment{Cache Global}.
					.4 proyectos.
						.5 miProyecto.
							.6 .ftt.
								.7 Plugins\DTcomment{Cache Local}.
					.4 \ldots.
			.2 \ldots.
	}
	\caption{Caches en un sistema Linux}
	\label{folders:plugins:linux}
\end{folders}

\begin{folders}[h]
	\dirtree{%
		.1 C:/.
			.2 Users.
				.3 username.
					.4 .ftt.
						.5 Plugins\DTcomment{Cache Global}.
					.4 proyectos.
						.5 miProyecto.
							.6 .ftt.
								.7 Pluginsitory\DTcomment{Cache Local}.
					.4 \ldots.
			.2 \ldots.
	}
	\caption{Caches en un sistema Windows}
	\label{folders:plugins:windows}
\end{folders}

\begin{folders}[h]
	\dirtree{%
		.1 /.
			.2 Users.
				.3 username.
					.4 .ftt.
						.5 Plugins\DTcomment{Cache Global}.
					.4 proyectos.
						.5 miProyecto.
							.6 .ftt.
								.7 Plugins\DTcomment{Cache Local}.
					.4 \ldots.
		.2 \ldots.
	}
	\caption{Caches en un sistema Mac OS X}
	\label{folders:plugins:osx}
\end{folders}

Para poder utilizar archivos de la cache de plugins, se requiere
que el nombre del archivo dado no contenga una ruta, ni relativa ni absoluta,
sino solamente el nombre del archivo .jar a utilizar. Como se puede observar,
las carpetas de \emph{Plugins} existen tanto de forma local como global, al
igual que la cache de dependencias. En el caso de especificar un archivo a cargar
\fronttier buscará primero en la carpeta actual, en caso de encontrarlo, utilizará
este, de lo contrario recurrirá a buscar en la cache de plugins local, y finalmente
en la cache de plugins global, para finalmente reportar un error en caso de no encontrar
el archivo en ninguna de las ubicaciones.\\

\subsection{Comandos de manejo de la aplicación}
\label{subsec:commands}

\fronttier incluye una serie de comandos adicionales para facilitar el manejo de
la herramienta y realizar tareas básicas sin necesidad de utilizar un archivo
de configuración, o realizar dichas tareas manualmente. Esto incluye cosas como
instalar dependencias, copiar dependencias a la cache, borrar dependencias de la
cache, instalar o desinstalar plugins, etc.

\subsubsection{Instalar dependencias}
\label{subsubsec:commands:install}

Es posible instalar dependencias sin la necesidad de declararlas en un archivo de
configuración mediante el uso del comando \emph{install}. Este comando tomará como
argumento el nombre de la dependencia y la instalará en la carpeta actual o la
especificada mediante argumentos especiales como \ddashcmd{destination}. Al igual
que cuando actúa leyendo de un \conffile, \fronttier buscará primero en las caches
y en caso de no encontrar la dependencia recurrirá a los repositorios en línea.
\cmd{fronttier install org.jquery:jquery:1.9.0}
A su vez, la mayoria de los argumentos especificados en la sección
\ref{subsec:guide:cli} pueden ser utilizados, quedando exceptuados aquellos
que refieren a los \conffiles.
\cmd{fronttier install org.jquery:jquery:1.9.0 -l --destination ./WEB-INF/html}

\subsubsection{Cachear dependencias}
\label{subsubsec:commands:cache}

El comando \emph{cache} y el comando \emph{delete} permiten guardar y borrar
dependencias en la cache de dependencias de \fronttier. Aquí, los
argumentos mostrados en \ref{subsubsec:guide:localmode} y el modo verbose son
las únicas opciones disponibles.
\cmd{fronttier cache org.jquery:jquery:1.9.0}
\cmd{fronttier remove org.jquery:jquery:1.9.0 -l}

\subsubsection{Instalar plugins}
\label{subsubsec:commands:plugins}

Instalar y desinstalar plugins debe hacerse manualmente o utilizando los comandos
\emph{plug} y \emph{unplug} respectivamente. Aquí también, solo tendrán sentido los
modificadores \ddashcmd{global} o \ddashcmd{local} y el modo verbose.
\cmd{fronttier plug myCoolPlugin}
\cmd{fronttier unplug myCoolPlugin}

Los plugins no cuentan con un repositorio, con lo cual, deben encontrarse en alguna
carpeta del sistema para poder hacer referencia al mismo.

\subsection{Archivos de configuración de la herramienta}
\label{subsec:guide:conffiles}

\fronttier trabaja con valores predeterminados, que incluyen cosas como el
formato del archivo de configuración a utilizar, los flags activos, etc.
Estos valores pueden ser modificados mediante los llamados archivos de
configuración de \fronttier.\\
Precisamente, pueden existir en determinadas ubicaciones del sistema, una serie
de archivos que deben tener por nombre \emph{.fttrc} y que modifican los valores
por defecto de la herramienta cuando esta es utilizada desde la linea de comandos.
Los archivos deben encontrarse en alguna de las siguientes 3 ubicaciones:
\begin{itemize}
	\setlength{\itemsep}{1pt}
	\setlength{\parskip}{0pt}
	\setlength{\parsep}{0pt}
	\item La carpeta de trabajo actual, o carpeta de proyecto
	\item La carpeta personal del usuario
	\item La carpeta de archivos de configuración, comúnmente denominada /etc
\end{itemize}
Basados en el ejemplo del usuario trabajando sobre el proyecto \emph{miProyecto}
dentro del directorio \emph{proyectos} en su carpeta personal, podemos ver la
ubicación de cada uno de estos archivos en los sistemas operativos más comunes
en los ejemplos \ref{folders:conf:linux}, \ref{folders:conf:windows} y
\ref{folders:conf:osx} de las estructuras de carpetas.
\begin{folders}[h]
	\dirtree{%
		.1 /.
			.2 etc.
				.3 .fttrc\DTcomment{Configuración Global}.
			.2 home.
				.3 username.
					.4 .fttrc\DTcomment{Configuración de Usuario}.
					.4 proyectos.
						.5 miProyecto
							.6 .fttrc\DTcomment{Configuración Local}.
					.4 \ldots.
			.2 \ldots.
	}
	\caption{Archivos de configuración en un sistema Linux}
	\label{folders:conf:linux}
\end{folders}

\begin{folders}[h]
	\dirtree{%
		.1 C:/.
			.2 Windows
				.3 System32
					.4 Drivers
						.5 etc.
							.6 .fttrc\DTcomment{Configuración Global}.
			.2 Users.
				.3 username.
					.4 .fttrc\DTcomment{Configuración de Usuario}.
					.4 proyectos.
						.5 miProyecto
							.6 .fttrc\DTcomment{Configuración Local}.
					.4 \ldots.
			.2 \ldots.
	}
	\caption{Archivos de configuración en un sistema Windows}
	\label{folders:conf:windows}
\end{folders}

\begin{folders}[h]
	\dirtree{%
		.1 /.
			.2 etc.
				.3 .fttrc\DTcomment{Configuración Global}.
			.2 Users.
				.3 username.
					.4 .fttrc\DTcomment{Configuración de Usuario}.
					.4 proyectos.
						.5 miProyecto
							.6 .fttrc\DTcomment{Configuración Local}.
					.4 \ldots.
			.2 \ldots.
	}  
	\caption{Archivos de configuración en un sistema Mac OS X}
	\label{folders:conf:osx}
\end{folders}
Cada archivo posee configuraciones para la ejecución por defecto de \fronttier.\\

\subsection{Distintos formatos del archivo configuración}
\label{subsec:formats}

Como ya se ha mencionado en el presente, existen múltiples formatos de archivos
de configuración posibles. El formato elegido usando el método expresado en la
sección \ref{subsubsec:guide:confformat} afecta así a la forma en la que deben
estar declaradas las dependencias en el archivo. En las secciones siguientes
se muestran los formatos incluidos por defecto con la herramienta.

\subsubsection{Formato de configuración estándar}
\label{subsubsec:formats:standar}

El formato de configuración estándar o formato \fronttier intenta ser el método
por defecto para declarar las dependencias y repositorios.\\
El documento posee dos secciones, de las cuales una es opcional y una
obligatoria. Cada sección se identifica con el nombre, y luego entre llaves
se declararán los elementos pertinentes de dicha sección. A continuación se muestra
un ejemplo de una declaración de este formato de archivo.
\begin{listing}[ht]
\begin{minted}[frame=lines]{javascript}
repositories {
  http://myrepository.com
  https://some-other-repo.org/repository
}

dependencies {
  org.jquery:jquery:2.1.1
  org.jquery:jquery-ui:1.11.0
  com.twitter:bootstrap:3.2.0
}
\end{minted}
\caption{Ejemplo del formato \emph{fronttier}}
\label{guide:fronttier:sample}
\end{listing}
El formato Backus-Naur Extendido para este archivo es
\begin{listing}[ht]
\begin{minted}[frame=lines]{javascript}
delimiter = ";"|"\n"
string = "\s", {"\s"}
repository	= string, delimiter
repositories = "repositories {" {repository} "}"
dependency	= string, ":", string, [":", string]
dependencies = "dependencies {" {dependency} "}"
depsfirst = dependencies, [repositories]
depslast = [repositories], dependencies
grammar = depsfirst | depslast
\end{minted}
\caption{EBNF de \emph{fronttier}}
\label{guide:fronttier:ebnf}
\end{listing}
Se puede apreciar que es posible no declarar repositorios adicionales, o declarar una
lista de dependencias vacía, así como también separar los componentes por un salto de
linea o por un punto y coma (;). Finalmente, los espacios en blanco no son tenidos en
cuenta, si bien esto no se refleja en la notación EBNF mostrada por simplicidad.
\begin{listing}[ht]
\begin{minted}[frame=lines]{javascript}
repositories {
  http://myrepository.com; https://some-other-repo.org/repository
}
dependencies {
  org.jquery:jquery:2.1.1
  org.jquery:jquery-ui:1.11.0
  com.twitter:bootstrap:3.2.0
}
\end{minted}
\caption{Otro ejemplo del formato \emph{fronttier}}
\label{guide:fronttier:sample_two}
\end{listing}
Este formato se encuentra registrado bajo el identificador \emph{fronttier}
y es el utilizado por la herramienta por defecto.

\subsubsection{Formato de configuración estilo \maven}
\label{subsubsec:formats:mvn}

Para los usuarios de \maven puede ser más cómodo o útil tener un formato que se adecue
a los archivos POM que utilizan. Por ello, el formato \emph{xml} permite configurar las
dependencias y repositorios utilizando XML. A continuación se muestra un ejemplo de dicho
formato

\begin{listing}[ht]
\begin{minted}[frame=lines]{xml}
<?xml version="1.0"?>
<!DOCTYPE fronttier-xml>
<fronttier>
  <repositories>
    <repository>http://myrepository.com</repository>
    <repository>https://some-other-repo.org/repository</repository>
  </repositories>
  <dependencies>
    <dependency>
      <group>org.jquery</group>
      <name>jquery</name>
      <version>2.11.0</version>
    </dependency>
    <dependency>
      <group>org.jquery</group>
      <name>jquery-ui</name>
    </dependency>
  </dependencies>
</fronttier>
\end{minted}
\caption{Ejemplo del formato \emph{xml}}
\label{guide:xml:sample}
\end{listing}
Aquí aplican las reglas básicas de escritura de documentos XML, las cuales
se encuentran dadas por el siguiente documento DTD
\begin{listing}[ht]
\begin{minted}[frame=lines]{xml}
<!ELEMENT fronttier (repositories?, dependencies)>
<!ELEMENT repositories (repository*)>
<!ELEMENT repository (#CDATA)>
<!ELEMENT dependencies (dependency*)>
<!ELEMENT dependency (group, name, version?)>
<!ELEMENT group (#CDATA)>
<!ELEMENT name (#CDATA)>
<!ELEMENT version (#CDATA)>
\end{minted}
\caption{DTD del formato \emph{attrxml}}
\label{guide:xml:dtd}
\end{listing}

\subsubsection{Formato de configuración estilo \ivy}
\label{subsubsec:formats:ivy}

Al igual que ocurre con los usuarios de \maven, los usuarios de \apache
\ivy pueden encontrar más practico utilizar un documento que sigue la
siguiente sintaxis para sus documentos

\begin{listing}[ht]
\begin{minted}[frame=lines]{xml}
<?xml version="1.0"?>
<!DOCTYPE fronttier-attrxml>
<fronttier>
  <repositories>
    <repository url="http://myrepository.com"/>
    <repository url="https://some-other-repo.org/repository"/>
  </repositories>	
  <dependencies>
    <dependency group="org.jquery" name="jquery" ver="2.11.0"/>
    <dependency group="org.jquery" name="jquery-ui"/>
  </dependencies>
</fronttier>
\end{minted}
\caption{Ejemplo del formato \emph{attrxml}}
\label{guide:attrxml:sample}
\end{listing}
En este caso, el DTD utilizado es el siguiente
\begin{listing}[ht]
\begin{minted}[frame=lines]{xml}
<!ELEMENT fronttier (repositories?, dependencies)>
<!ELEMENT repositories (repository*)>
<!ELEMENT repository EMPTY>
<!ATTLIST repository url CDATA #REQUIRED>
<!ELEMENT dependencies (dependency*)>
<!ELEMENT dependency EMPTY>
<!ATTLIST dependency group CDATA #REQUIRED>
<!ATTLIST dependency name CDATA #REQUIRED>
<!ATTLIST dependency ver CDATA #IMPLIED>
\end{minted}
\caption{DTD del formato \emph{attrxml}}
\label{guide:attrxml:dtd}
\end{listing}
\subsection{Integración en sistemas}
\label{subsec:guide:systems}

\subsubsection{Integrando desde \scala}
\label{subsubsec:guide:systems:scala}

\subsubsection{Integrando desde \java}
\label{subsubsec:guide:system:java}