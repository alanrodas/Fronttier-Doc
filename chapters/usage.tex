\section{Uso de la herramienta}
\label{sec:guide}

La presente sección muestra el uso de la herramienta desarrollada. Inicialmente 
se verá su uso mediante la \cli. Luego, se mostrará la forma en la que la 
herramienta puede ser usada y extendida desde otros lenguajes.\\


\subsection{Uso desde la \cli}
\label{subsec:guide:cli}

Para correr la aplicación desde la \cli es necesario ejecutar el archivo \jar
de la herramienta, a través de la \jvm. El siguiente código muestra como 
ejecutar \fronttier desde la interfaz de \cli:
\cmd{java -jar fronttier.jar}
Adicionalmente, se puede pasar una serie de argumentos a la aplicación para 
configurarla a gusto y necesidad del usuario, tal como se muestra a 
continuación:
\cmd{java -jar fronttier.jar <argumentos>}
Con intención de simplificar el trabajo del usuario al momento de ejecutar la
aplicación, \fronttier incluye un programa que puede ser colocado en 
cualquier lugar accesible para el usuario\footnote{
	Un lugar accesible es, por ejemplo, cualquier carpeta que se encuentre 
	dentro de la variable de entorno \quoted{PATH} en el sistema operativo del 
	usuario.
}, permitiendo ejecutar la aplicación de forma directa, tal como se muestra en 
el siguiente ejemplo:
\cmd{fronttier <argumentos>}
A continuación se verán los argumentos posibles en más detalle, así como la 
forma de configuración de la herramienta.

\subsubsection{Uso con los valores por defecto}
\label{subsubsec:guide:defaults}

Es posible llamar al programa sin ningún argumento y obtener funcionalidad del 
mismo mediante convenciones que \fronttier utiliza.\\
Al ser ejecutada sin ningún tipo de argumentos, \fronttier utilizará los 
valores por defecto de la aplicación. Éstos incluyen: utilizar la \cacheg, 
buscar \dependencies en un archivo de configuración con el formato estándar de 
\fronttier con el nombre \code{fronttier.ftt} e instalar las \dependencies 
descargadas en el directorio local.\\

\subsubsection{Selección del archivo de configuración y su formato}
\label{subsubsec:guide:confformat}

Por defecto, \fronttier intenta localizar en el directorio actual un archivo 
con el nombre de \texttt{fronttier.ftt}. El mismo es donde se declaran las 
\dependencies y repositorios del \project.\\
A su vez, el usuario puede especificar un archivo distinto en donde buscar 
\dependencies. Ésto se logra mediante la opción \ddashcmd{filename} o 
\dashcmd{F} pasada como argumento a la aplicación e incluyendo luego el nombre 
del archivo que se desea utilizar para declaración de \dependencies y 
repositorios. A continuación se puede observar un ejemplo de cómo llamar a la 
aplicación para que lea un archivo llamado \code{dependencies.ftt}:
\cmd{fronttier --filename dependencies.ftt}
Los archivos mencionados pueden tener internamente distintos formatos, es 
decir, distintas formas de representar la información. Cada formato posee un 
nombre único por el cual el usuario puede identificarlo y, así, indicarle a la 
herramienta qué debería esperar de los contenidos del archivo de configuración. 
Esto se logra mediante el argumento \ddashcmd{config} o \dashcmd{c}. El 
siguiente código muestra como elegir un formato con nombre \code{xml}:
\cmd{fronttier --config xml}
Los formatos registrados por defecto en \fronttier son:
\begin{itemize}
	\setlength{\itemsep}{1pt}
	\setlength{\parskip}{0pt}
	\setlength{\parsep}{0pt}
	\item fronttier
	\item xml
	\item attrxml
\end{itemize}
Sin embargo, el usuario puede registrar nuevos formatos mediante extensiones
a la herramienta. En caso de no especificar el formato del archivo, el formato 
\emph{fronttier} es el utilizado.\\
El formato elegido del archivo repercute sobre el nombre por defecto del
mismo. Por ejemplo, el nombre de archivo por defecto para el formato \code{xml}
es \code{fronttier.xml}, al igual que para el formato \emph{attrxml}. Por su 
parte, el formato \code{fronttier} buscará un archivo por el nombre de 
\code{fronttier.ftt}.\\
Los argumentos \ddashcmd{config} y \ddashcmd{filename} se pueden usar al mismo
tiempo dando lugar a cualquier combinación de nombre de archivo y formato.
Por ejemplo, el siguiente código muestra cómo ejecutar \fronttier para que 
utilice un archivo \code{dependencies.xml} que posee formato \code{xml}.
\cmd{fronttier --config xml --filename dependencies.xml}
La estructura interna de cada uno de los formatos, así como la forma de 
declarar \dependencies y repositorios es especificada en la sección 
\ref{subsec:formats}.

\subsubsection{Elección de carpeta de destino}
\label{subsubsec:guide:folder}

Por defecto \fronttier trabaja sobre la carpeta actual, y es allí donde se 
\emph{instalan} las \dependencies descargadas.\\
No obstante, la carpeta sobre la que trabajará la herramienta puede ser 
especificada, de forma tal que las \dependencies se instalen en alguna carpeta
específica o una subcarpeta de la carpeta local. Para ello se utiliza el  
comando \ddashcmd{destination} o \dashcmd{d} seguido de la ubicación en donde 
se desean descargar las \dependencies.
\cmd{fronttier --destination <folder>}
El valor por defecto es \code{.} (punto), es decir, la carpeta actual. Por 
tanto, los siguientes dos comandos son equivalentes:
\cmd{fronttier}
\cmd{fronttier --destination .}
La elección de la carpeta de destino resulta útil en proyectos donde el código 
compilado termina en una carpeta específica que no forma parte del proyecto y 
que se elimina cada vez que se recompila el mismo. Así, o más habitual es 
elegir  un subdirectorio de la carpeta de trabajo donde se deben colocar las 
\dependencies previo a ejecutar el sistema. El siguiente ejemplo muestra cómo 
indicarle a \fronttier que utilice como carpeta de descarga un subdirectorio 
con el nombre de \quoted{WEB-INF/html}:
\cmd{fronttier --destination ./WEB-INF/html}
Si bien no es algo común, se pueden especificar también direcciones absolutas 
en lugar de relativas. En caso de que las direcciones contengan espacios se 
debe utilizar comillas simples o dobles para especificar la dirección completa, 
como se muestra en el siguiente ejemplo:
\cmd{fronttier --destination "C:\Web Dependencies"}
Se debe tener en cuenta que \fronttier no realiza ninguna acción para limpiar 
la carpeta de destino, por lo que el usuario deberá garantizar su limpieza 
previa a futuras ejecuciones.

\subsubsection{Cache local y cache global}
\label{subsubsec:guide:localmode}

\fronttier mantiene en la máquina del usuario una cache con todos los archivos 
que va descargando a lo largo de su uso. Ésto permite acelerar el proceso de 
\emph{instalación} de las \dependencies en un proyecto particular. Así, 
\fronttier mantiene dos lugares que utiliza como cache, la \cachel y la 
\cacheg.\\
La \cachel tiene alcance sólo sobre el proyecto de código sobre el cual se está 
trabajando (es decir, el \conffile que se está leyendo y el directorio sobre el 
cual se están instalando las \dependencies). Su uso se limita a evitar tener 
que descargar las \dependencies en subsecuentes ejecuciones del programa en el 
mismo proyecto.\\
Por su parte, la \cacheg contiene las \dependencies descargadas por todos los 
proyectos. Ésta cache se utiliza, no sólo para evitar la descarga de 
\dependencies en futuras ejecuciones del programa sobre un proyecto 
determinado, sino además para evitar la descarga en caso de que dos proyectos 
independientes requieran eventualmente la misma \dependency.\\
Las caches permiten adicionalmente trabajar sin la necesidad de una conexión a 
Internet, ya que no se deben descargar las \dependencies, pues ya se encuentran 
el equipo.\\
La \cacheg se encuentra en una carpeta creada por \fronttier en el directorio 
personal del usuario. Por su parte, la \cachel se encuentra en una carpeta 
similar, pero localizada en el directorio del proyecto sobre el que aplica 
dicha cache. Las caches se ubican en los siguientes directorios.

\begin{description}
	\item[Cache global] \hfill \\
	$\sim$/.ftt/Repository
	\item[Caches local] \hfill \\
	\%PROJECT\_FOLDER\%/.ftt/Repository
\end{description}

Donde \quoted{\%PROJECT\_FOLDER\%} es el directorio del proyecto sobre el cual 
se está trabajando.\\
Para comprender mejor el sistema de caches es necesario comprender cómo 
\fronttier descarga \dependencies. En primer lugar, \fronttier leerá el 
\conffile y analizará las \dependencies a instalar. Para cada \dependency, se 
evaluará si existe en la \cachel. Si se encuentra, se copiará desde la misma a 
la carpeta donde se requiera instalar la \dependency. En caso de no 
encontrarse, se recurrirá a buscarla en la \cacheg para copiarla desde allí. 
Finalmente, si no se encuentra en las caches, se descargará de alguno de los 
\onlinerepo disponibles. \fronttier indicará un error al usuario si no 
encuentra en éstos últimos la \dependency.\\
\fronttier copiará por defecto las \dependencies que descargue desde 
\onlinerepo a la \cacheg para futuros usos. Éste comportamiento está dado 
mediante el \flag \ddashcmd{global} o \dashcmd{g}. Así, los siguientes dos 
códigos son equivalentes:
\cmd{fronttier}
\cmd{fronttier -g}
Mediante el \flag \ddashcmd{local} o \dashcmd{l} se puede indicar que no se 
desea copiar las \dependencies a la \cacheg sino a la local. Esto puede 
resultar útil en algunos proyectos. El \flag \ddashcmd{nocache} evita que se 
guarden las \dependencies descargadas en cualquiera de las caches.\\
Por último, se puede utilizar el \flag \dashcmd{f} o \ddashcmd{force} para 
forzar la descarga de una \dependency desde los \onlinerepo incluso si
se encuentran en alguna cache. Esta acción reemplazará la versión actual en
la cache por la recién descargada.\\
Todos estos \flags pueden combinarse para dar lugar a acciones determinadas. 
Por ejemplo, el siguiente código muestra cómo descargar \dependencies sin 
guardarlas en la cache, y utilizando siempre la versión disponible en los 
\onlinerepo.
\cmd{fronttier --nocache -f}
También es posible descargar las \dependencies en la \cachel, 
independientemente de si se encuentran en alguna de las caches, como muestra el 
siguiente código.
\cmd{fronttier -lf}
Cabe destacar que los \flags en su formato corto pueden agruparse utilizando un
guión solo (-) y pasando todos los \flags juntos. Así \dashcmd{lf} es 
equivalente a \dashcmd{l} \dashcmd{f}.


\subsubsection{Modo verbose}
\label{subsubsec:guide:verbose}

Por defecto \fronttier brinda muy poca información al usuario con respecto a
las acciones que realiza en cada momento, indicando solamente los errores que 
se puedan llegar a producir. Sin embargo, es posible cambiar dicho 
comportamiento activando el modo \code{verbose}.\\
Precisamente el \flag \ddashcmd{verbose} o \dashcmd{v} activa este modo 
permitiendo ver información de todos los pasos que la herramienta va 
realizando. Esto resulta útil en caso de errores inesperados para poder 
descubrir el origen de los mismos.\\


\subsubsection{Agregar archivos de expansión}
\label{subsubsec:guide:expand}

La herramienta \fronttier puede ser extendida por código del usuario. Esto 
permite, entre otras cosas, agregar nuevos formatos de \conffile.\\
Para extender el programa, basta cargar los archivos \jar (código compilado a
\bytecode \java) que contienen las extensiones, a los cuales llamaremos 
\plugins. Ésto se realiza mediante el argumento \ddashcmd{load} o \dashcmd{L} 
el cual recibe el o los \plugins a cargar. Por ejemplo, para cargar los 
\plugins \quoted{first.jar} y \quoted{second.jar} se debería ejecutar la 
aplicación de la siguiente forma:
\cmd{fronttier -L first second}
La extensión del archivo no es necesaria pues se asume que siempre será .jar.\\
Los archivos de extensión, en este caso, se encuentran en la carpeta actual
desde donde se ejecuta la aplicación. También se pueden utilizar rutas
absolutas o relativas para llamar a los archivos como se muestra a continuación:
\cmd{fronttier --load "C:\Fronttier Extensions\first"}
Finalmente, cabe destacar la posibilidad de \emph{instalar} extensiones en
una carpeta global o local para que puedan ser utilizados en cualquier
momento. Esta carpeta actúa de forma similar a la cache para los archivos
de \dependency. De hecho, su ubicación es precisamente en un directorio hermano
a la carpeta del repositorio de cache. A continuación se detalla su ubicación:

\begin{description}
	\item[Cache de plugins global] \hfill \\
	$\sim$/.ftt/Plugins
	\item[Caches de plugins local] \hfill \\
	\%PROJECT\_FOLDER\%/.ftt/Plugins
\end{description}

Donde \quoted{PROJECT\_FOLDER} es el directorio del proyecto sobre el cual 
se está trabajando.\\
Los archivos \jar que se encuentren en una carpeta de \emph{cache de plugins} 
pueden  ser utilizados como si se encontraran en la carpeta actual, es decir, 
mediante su nombre.\\
Cuando \fronttier recibe un \plugin a cargar que no contiene ruta, primero 
evaluará su existencia en la carpeta actual, luego en la \plugincachel, y 
finalmente en la \plugincacheg. Utilizará entonces aquel que  encuentre 
primero, indicando un error en caso de no encontrar coincidencias.\\

\subsection{Comandos de manejo de la aplicación}
\label{subsec:commands}

\fronttier incluye una serie de comandos adicionales para facilitar el manejo de
la herramienta y realizar tareas básicas sin necesidad de utilizar un archivo
de configuración o realizar dichas tareas manualmente. Esto incluye cosas como
instalar \dependencies, copiar \dependencies a la cache, borrar \dependencies 
de la cache, instalar o desinstalar \plugins, etc.

\subsubsection{Instalación de \dependencies}
\label{subsubsec:commands:install}

Es posible instalar \dependencies sin la necesidad de declararlas en un 
\conffile mediante el uso del comando \code{install}. Este comando tomará como 
argumento el nombre de la \dependency a instalar. El siguiente código muestra 
cómo funciona el comando:\\
\cmd{fronttier install {company}:{package}:[{version}]}
Donde \code{company} y \code{package} son cadenas de texto que hacen referencia 
inequívoca a la \dependency a instalar. Opcionalmente \code{version} puede ser 
especificado para indicar una versión puntual del mismo (si no se indica, se 
descargará la más reciente). El siguiente código muestra un ejemplo 
puntual para descargar la biblioteca \code{jQuery} en su versión \code{1.9.0}:
\cmd{fronttier install org.jquery:jquery:1.9.0}
La mayoría de los argumentos descriptos en la sección \ref{subsec:guide:cli} 
pueden ser utilizados, quedando exceptuados aquellos que refieren a los  
\conffiles. Por ejemplo, el argumento \ddashcmd{destination} o \dashcmd{d}, 
especificados en la sección \ref{subsubsec:guide:folder}, puede ser utilizado 
como se muestra a continuación:
\cmd{fronttier install org.jquery:jquery:2.0 -l -d ./WEB-INF/html}
La búsqueda del archivo en las caches es exactamente igual a cuando se trabaja 
con un \conffile. 

\subsubsection{Cachear \dependencies}
\label{subsubsec:commands:cache}

El comando \code{cache} y el comando \code{uncache} permiten guardar y borrar
\dependencies en la cache de \dependencies de \fronttier. Se puede entonces pedir 
que un determinado archivo se descargue a la cache de la siguiente forma:\\
\cmd{fronttier cache org.jquery:jquery:1.9.0}
Los argumentos \ddashcmd{global} o \dashcmd{g} y \ddashcmd{local} o 
\dashcmd{l}, presentados en la sección \ref{subsubsec:guide:localmode}, pueden 
ser utilizados para indicar exactamente sobre qué cache se va a trabajar. Por 
ejemplo, el siguiente código remueve un archivo de la \cachel:
\cmd{fronttier uncache org.jquery:jquery:1.9.0 --local}

\subsubsection{Instalar \plugins}
\label{subsubsec:commands:plugins}

Los \plugins pueden ser instalados a las \emph{caches de plugins} mediante los 
comandos \code{plug} y desinstalados con el comando \code{unplug}. Los \flags
\ddashcmd{global} o \dashcmd{g} y \ddashcmd{local} o \dashcmd{l}, presentados 
en la sección \ref{subsubsec:guide:localmode}, pueden ser utilizados. A 
continuación se muestra cómo instalar un \plugin desde la carpeta local a la 
\plugincacheg:
\cmd{fronttier plug myCoolPlugin}
Y aquí cómo desinstalar un \plugin desde la \plugincachel:
\cmd{fronttier unplug myCoolPlugin -l}

\subsection{Archivos de defaults de la herramienta}
\label{subsec:guide:conffiles}

\fronttier trabaja con varios valores predeterminados, como el formato del 
\conffile a utilizar o qué cache usar. Éstos valores pueden ser modificados 
mediante los llamados archivos de defaults de \fronttier.\\
Así, pueden existir una serie de archivos que deben tener por nombre 
\emph{defaults} y que modifican los valores por defecto de la herramienta 
cuando ésta es utilizada desde la \cli. Éstos archivos se 
encuentran en:

\begin{description}
	\item[Archivo de \defaults global] \hfill \\
	$\sim$/.ftt/defaults
	\item[Archivo de \defaults local] \hfill \\
	\%PROJECT\_FOLDER\%/.ftt/defaults
\end{description}

Donde \quoted{\%PROJECT\_FOLDER\%} es el directorio del proyecto sobre el cual 
se está trabajando.\\
Cada archivo posee configuraciones para la ejecución por defecto de 
\fronttier. En primer lugar, se cargarán los valores por defecto
generales, luego se leerán los valores del archivo \defaults global. En este 
momento, los valores definidos en ese archivo \quoted{sobreescribirán}\footnote{
	Es decir, se dejará el valor por defecto para todas aquellos atributos a 
	los cuales no se les haya asignado un nuevo valor en el archivo \defaults, 
	pero en caso de que se haya declarado un valor, se utilizará este último, 
	descartando el valor anterior.
} a los valores por defecto. Finalmente, se leerá el archivo de \defaults 
local, \quoted{sobreecribiendo} los valores que hayan resultado de leer el archivo de \defaults global.\\
\fronttier se distribuye con un manual de uso que provee las indicaciones 
necesarias para crear y modificar éstos archivos, así como los contenidos 
esperados en los mismos.


\subsection{Distintos formatos del archivo configuración}
\label{subsec:formats}

Como ya se ha mencionado en el presente documento, existen múltiples formatos 
de archivos de configuración posibles. El formato elegido usando el método 
expresado en la sección \ref{subsubsec:guide:confformat} influye así en la 
forma en la que deben estar declaradas las \dependencies en el archivo. En las 
secciones siguientes se muestran los formatos incluidos por defecto con la 
herramienta.

\subsubsection{Formato de configuración estándar}
\label{subsubsec:formats:standar}

El formato de configuración estándar o formato \fronttier es el método por 
defecto para declarar las \dependencies y repositorios.\\
En este formato se deben declarar dos secciones: \code{repositories} y 
\code{dependencies}. La sección \code{repositories}, la cual es opcional, 
indica los repositorios adicionales que se desean utilizar, uno por línea, o 
separados por punto y coma. La sección \code{dependencies} es obligatoria e 
indica las \dependencies requeridas por el proyecto. Las secciones se 
encuentran 
delimitadas por llaves, y sólo puede haber una sección 
\code{dependencies} y una sección \code{repositories}.\\
En la sección \code{dependencies} cada \dependency es declarada en una nueva 
línea (o separándolas por punto y coma) indicando la \code{company} seguida de 
dos puntos y el nombre (\code{name}). De forma opcional, se puede agregar dos 
puntos luego del nombre e indicar la versión puntual que desea instalar.\\
En este formato de archivo, los espacios en blanco no son tenidos en cuenta y 
se pueden agregar comentarios de línea mediante el caracter \#.\\
El código \ref{guide:fronttier:sample} muestra un ejemplo de una declaración de 
este formato de archivo.

\begin{listing}[ht]
	\begin{minted}[frame=lines]{javascript}
repositories {
  http://myrepository.com
  https://some-other-repo.org/repository
}

dependencies {
  org.jquery:jquery:2.1.1; org.jquery:jquery-ui:1.11.0
  # Ésto es un comentario
  com.twitter:bootstrap:3.2.0
}
	\end{minted}
	\caption{Ejemplo del formato \emph{fronttier}}
	\label{guide:fronttier:sample}
\end{listing}

El formato Backus-Naur Extendido para éste archivo se muestra de forma 
simplificada en el código \ref{guide:fronttier:ebnf}.

\begin{listing}[ht]
	\begin{minted}[frame=lines]{javascript}
string         = "\s", {"\s"}
repository     = [string "::"], string
repositories   = "repositories {" {repository} "}"
dependency     = string, "::", string, ["::", string]
dependencies   = "dependencies {" {dependency} "}"
depsfirst      = dependencies, [repositories]
depslast       = [repositories], dependencies
grammar        = depsfirst | depslast
	\end{minted}
	\caption{EBNF de \emph{fronttier}}
	\label{guide:fronttier:ebnf}
\end{listing}

Éste formato se encuentra registrado bajo el identificador \emph{fronttier}.

\subsubsection{Formato de configuración estilo \maven}
\label{subsubsec:formats:mvn}

El formato \xml permite configurar las \dependencies y repositorios utilizando 
\xml, que puede ser más cómodo para los usuarios de \maven. En este 
formato los repositorios, \dependencies y todas las propiedades de cada una de 
las \dependencies se expresan como texto entre tags \xml anidados.\\
Los tags posibles y sus anidaciones están dadas por un archivo DTD, el cual se 
puede apreciar en el código \ref{guide:xml:dtd}.

\begin{listing}[ht]
	\begin{minted}[frame=lines]{xml}
<!ELEMENT fronttier (repositories?, dependencies)>
<!ELEMENT repositories (repository*)>
<!ELEMENT repository (type?, url)>
<!ELEMENT type (#PCDATA)>
<!ELEMENT url (#PCDATA)>
<!ELEMENT dependencies (dependency*)>
<!ELEMENT dependency (group, name, version?)>
<!ELEMENT group (#PCDATA)>
<!ELEMENT name (#PCDATA)>
<!ELEMENT version (#PCDATA)>
	\end{minted}
	\caption{DTD del formato \emph{xml}}
	\label{guide:xml:dtd}
\end{listing}

Los nombres y las convenciones utilizadas en el formato se corresponden 
completamente con aquellos usados por \apache \maven para una más fácil 
adopción por parte de los usuarios de esta herramienta. En el código 
\ref{guide:xml:sample} se muestra un ejemplo del formato \code{xml}.

\begin{listing}[ht]
	\begin{minted}[frame=lines]{xml}
<?xml version="1.0"?>
<!DOCTYPE fronttier PUBLIC "-/Fronttier/xml.dtd"
    "http://fronttier.alanrodas.com/dtd/xml.dtd">
<fronttier>
  <repositories>
    <repository>
      <url>http://myrepository.com</url>
    </repository>
    <repository>
      <type>git</type>
      <url>https://some-other-repo.org/repository</url>
    </repository>
  </repositories>
  <dependencies>
    <dependency>
      <group>org.jquery</group>
      <name>jquery</name>
      <version>2.11.0</version>
    </dependency>
    <dependency>
      <group>org.jquery</group>
      <name>jquery-ui</name>
    </dependency>
  </dependencies>
</fronttier>
	\end{minted}
	\caption{Ejemplo del formato \emph{xml}}
	\label{guide:xml:sample}
\end{listing}

\subsubsection{Formato de configuración estilo \ivy}
\label{subsubsec:formats:ivy}

Además del formato \code{xml}, el formato \code{attrxml} también utiliza \xml 
para declarar \dependencies. A diferencia del anterior, éste es más 
compacto, ya que los valores se declaran como atributos de los tags \xml.\\
Dicho formato respeta las convenciones impuestas por \apache \ivy, pudiendo ser 
de interés para los usuarios de la herramienta.\\
La estructura de éstos archivos está dada por el DTD que se muestra en el código
\ref{guide:attrxml:dtd}

\begin{listing}[ht]
	\begin{minted}[frame=lines]{xml}
<!ELEMENT fronttier (repositories?, dependencies)>
<!ELEMENT repositories (repository*)>
<!ELEMENT repository EMPTY>
<!ATTLIST repository type CDATA #IMPLIED>
<!ATTLIST repository url CDATA #REQUIRED>
<!ELEMENT dependencies (dependency*)>
<!ELEMENT dependency EMPTY>
<!ATTLIST dependency group CDATA #REQUIRED>
<!ATTLIST dependency name CDATA #REQUIRED>
<!ATTLIST dependency ver CDATA #IMPLIED>
	\end{minted}
	\caption{DTD del formato \emph{attrxml}}
	\label{guide:attrxml:dtd}
\end{listing}

El código en \ref{guide:attrxml:sample} muestra un ejemplo del formato 
descripto.

\begin{listing}[ht]
	\begin{minted}[frame=lines]{xml}
<?xml version="1.0"?>
<!DOCTYPE fronttier PUBLIC "-/Fronttier/xmlattr.dtd"
    "http://fronttier.alanrodas.com/dtd/xmlattr.dtd">
<fronttier>
  <repositories>
    <repository url="http://myrepository.com"/>
    <repository url="https://some-other-repo.org/repository"/>
  </repositories>	
  <dependencies>
    <dependency group="org.jquery" name="jquery" ver="2.11.0"/>
    <dependency group="org.jquery" name="jquery-ui"/>
  </dependencies>
</fronttier>
	\end{minted}
	\caption{Ejemplo del formato \emph{attrxml}}
	\label{guide:attrxml:sample}
\end{listing}


\subsection{Integración en sistemas}
\label{subsec:guide:systems}

\fronttier no es solamente una herramienta que corre desde la \cli, sino que, 
por ser un archivo \jar, puede ser utilizado como una biblioteca de la  
plataforma \java. Ésto permite que se utilice desde otros programas escritos 
para cualquiera de los lenguajes que corren sobre la \jvm.\\
Para cada acción que se puede ejecutar desde la \cli, \fronttier provee objetos 
y métodos que pueden ser utilizados desde el código del usuario. Esta es la 
forma en que se pueden crear extensiones para la herramienta, como por ejemplo 
nuevos formatos de \conffiles.\\
El código de \fronttier es compatible por defecto con \scala, por estar hecho 
en dicho lenguaje. También se proveen bindings para \java que permiten utilizar 
todas las funcionalidades.\\
La mayoría de los lenguajes que corren desde la \jvm permiten importar código 
\java y utilizarlo. Sin embargo, el código \java no necesariamente se adapta a 
las técnicas de programación que se utilizan en ese lenguaje. Por eso, nuevos 
bindings que permitan usar la herramienta de forma transparente en estos 
lenguajes pueden ser creados a futuro.\\
En particular, el paquete \jcode{com.alanrodas.fronttier} provee una serie de 
traits, objetos y clases para extender la aplicación o usar sus 
características. A su vez, el paquete \jcode{com.alanrodas.fronttier.java} 
provee los bindings correspondientes para realizar las mismas acciones desde 
\java. La documentación completa se distribuirá oportunamente junto con la 
aplicación y podrá ir variando con el tiempo en base a nuevas versiones del 
programa.