%%%%%% Title definitions %%%%%%
\define{\worktitle}{Trabajo de Inserción Profesional}
\renewcommand{\title}{Fronttier}
\define{\subtitle}{\xmakefirstuc{manejador de dependencias} de la \viewtier para plataformas basadas en lenguajes para la \acl{jvm}}
\renewcommand{\author}{Alan Rodas Bonjour}
\define{\email}{alanrodas@gmail.com}
\define{\webpage}{http://alanrodas.com}
\define{\projecturl}{http://alanrodas.com/fronttier}
\define{\director}{Dra. Gabriela B. Arévalo}
\define{\directoremail}{garevalo@unq.edu.ar}
\renewcommand{\date}{\today}

%%%%%% Keywords %%%%%%
\define{\internet}{\emph{Internet}}
\define{\dependency}{\emph{dependencia}}
\define{\dependencies}{\emph{dependencias}}
\define{\depmgr}{\emph{manejador de dependencias}}
\define{\depmgrs}{\emph{manejadores de dependencias}}
\define{\clientserver}{\emph{cliente-servidor}}
\define{\conffile}{\emph{archivo de configuración}}
\define{\conffiles}{\emph{archivos de configuración}}

%%%%%% English words %%%%%%
\define{\mainframe}{\emph{mainframe}\xspace}
\define{\mainframes}{\emph{mainframes}\xspace}
\define{\framework}{\emph{framework}}
\define{\frameworks}{\emph{frameworks}}
\define{\toolkits}{\emph{toolkits}}
\define{\plugins}{\emph{plugins}}
\define{\bytecode}{\emph{bytecode}}

%%%%%% Acronyms %%%%%%
\define{\jvm}{\emph{\ac{jvm}}}
\define{\ria}{\emph{\acs{ria}}}
\define{\rias}{\emph{\acsp{ria}}}
\define{\cdn}{\emph{\ac{cdn}}}
\define{\cdns}{\emph{\acsp{cdn}}}

%%%%%% Languages %%%%%%
\define{\java}{\emph{Java}}
\define{\scala}{\emph{Scala}}
\define{\groovy}{\emph{Groovy}}
\define{\clojure}{\emph{Clojure}}
\define{\html}{\emph{\acs{html}}}
\define{\css}{\emph{\acs{css}}}
\define{\js}{\emph{JavaScript}}
\define{\htmlv}{\emph{\acs{html5}}}
\define{\servlets}{\emph{Servlets}}
\define{\xml}{\emph{\acs{xml}}}

%%%%%% Tools %%%%%%
\define{\sbt}{\emph{SBT}}
\define{\git}{\emph{\gls{git}}
\define{\svn}{\emph{\gls{svn}}}
\define{\apache}{\emph{Apache}}
\define{\maven}{\emph{Maven}}
\define{\ivy}{\emph{Ivy}}
\define{\opensource}{\emph{Open Source}}

%%%%%% Tiers %%%%%%
\define{\logic}{\emph{lógica de negocios}}
\define{\logictier}{\emph{capa de lógica de negocios}}
\define{\view}{\emph{presentación}\xspace}
\define{\viewtier}{\emph{capa de presentación}}
\define{\data}{\emph{datos}}
\define{\datatier}{\emph{capa de datos}}
